@c -*-texinfo-*-

@c  Copyright (c) 2018 Blake McBride
@c  All rights reserved.
@c
@c  Redistribution and use in source and binary forms, with or without
@c  modification, are permitted provided that the following conditions are
@c  met:
@c
@c  1. Redistributions of source code must retain the above copyright
@c  notice, this list of conditions and the following disclaimer.
@c
@c  2. Redistributions in binary form must reproduce the above copyright
@c  notice, this list of conditions and the following disclaimer in the
@c  documentation and/or other materials provided with the distribution.
@c
@c  THIS SOFTWARE IS PROVIDED BY THE COPYRIGHT HOLDERS AND CONTRIBUTORS
@c  "AS IS" AND ANY EXPRESS OR IMPLIED WARRANTIES, INCLUDING, BUT NOT
@c  LIMITED TO, THE IMPLIED WARRANTIES OF MERCHANTABILITY AND FITNESS FOR
@c  A PARTICULAR PURPOSE ARE DISCLAIMED. IN NO EVENT SHALL THE COPYRIGHT
@c  HOLDER OR CONTRIBUTORS BE LIABLE FOR ANY DIRECT, INDIRECT, INCIDENTAL,
@c  SPECIAL, EXEMPLARY, OR CONSEQUENTIAL DAMAGES (INCLUDING, BUT NOT
@c  LIMITED TO, PROCUREMENT OF SUBSTITUTE GOODS OR SERVICES; LOSS OF USE,
@c  DATA, OR PROFITS; OR BUSINESS INTERRUPTION) HOWEVER CAUSED AND ON ANY
@c  THEORY OF LIABILITY, WHETHER IN CONTRACT, STRICT LIABILITY, OR TORT
@c  (INCLUDING NEGLIGENCE OR OTHERWISE) ARISING IN ANY WAY OUT OF THE USE
@c  OF THIS SOFTWARE, EVEN IF ADVISED OF THE POSSIBILITY OF SUCH DAMAGE.

@node Command Line Utility

@html
@include style.css
@end html

@chapter Command Line Utility

In addition to utilizing @emph{Kiss} as a web application development
system, @emph{Kiss} also provides a command line interface.  This interface
allows you to build quick but powerful utilities to do things like updating
a database, parsing a CSV file, interfacing with a third-party REST service,
and more.  Basically, all of @emph{Kiss} is available except the @emph{Kiss}
REST server.

Although this system supports @emph{PostgreSQL} out-of-the-box, it
also supports any of the other databases with a slightly more complex
command line.

@section Building The Utility

The build system (@emph{bld}) has the ability to build a @emph{JAR}
file named @emph{KissGP.jar}.  The @emph{GP} stands for @emph{Groovy}
and @emph{PostgreSQL}.  Basically, it is a @emph{JAR} file capable of
running @emph{Groovy} scripts in the context of all of the @emph{Kiss}
utilities including access to a @emph{PostgreSQL} database.

(It should be noted that although @emph{Groovy} scripts are text /
source files, they nevertheless run at full compiled speed because
they are compiled at runtime.)

To build the @emph{JAR} file, type the following:

@example
./bld KissGP
@end example

A file named @emph{KissGP.jar} will be created in the @emph{build.work}
directory.  That, in addition to the @emph{JDK}, is all that is needed to
use the system.

@section Using The System

To use the @emph{Kiss} command line interface you first create the
@emph{Groovy} program you would like to run.  It may use all of the
@emph{Groovy} and @emph{Kiss} API.  For example, let's start with
something simple.  Create the following file named
@emph{test1.groovy}:

@example
static void main(String [] args) @{
    println "Hello world!"
@}
@end example

You can then run the program as follows:

@example
java -jar KissGP.jar test1
@end example

@emph{test1.groovy} can be extended arbitrarily to perform any function needed.

@section Databases Other Than PostgreSQL

Although @emph{KissGP.jar} comes bundled with support for the
@emph{PostgreSQL} database, @emph{KissGP.jar} can support databases
other than @emph{PostgreSQL} by adding the driver for the database and
using a slightly more complex command line as follows.

To use @emph{KissGP.jar} with Microsoft SQL Server, for example, you'd
have to include the database driver for it in addition to the
@emph{KissGP.jar} file and the file with your program.  The command
line would look as follows:

@example
java -cp mssql-jdbc-8.2.0.jre8.jar -jar KissGP.jar test1
@end example

The same is true of the other databases.  Please note that the drivers
you'll need are already in the @emph{libs} directory.


