@c -*-texinfo-*-

@c  Copyright (c) 2018 Blake McBride
@c  All rights reserved.
@c
@c  Redistribution and use in source and binary forms, with or without
@c  modification, are permitted provided that the following conditions are
@c  met:
@c
@c  1. Redistributions of source code must retain the above copyright
@c  notice, this list of conditions and the following disclaimer.
@c
@c  2. Redistributions in binary form must reproduce the above copyright
@c  notice, this list of conditions and the following disclaimer in the
@c  documentation and/or other materials provided with the distribution.
@c
@c  THIS SOFTWARE IS PROVIDED BY THE COPYRIGHT HOLDERS AND CONTRIBUTORS
@c  "AS IS" AND ANY EXPRESS OR IMPLIED WARRANTIES, INCLUDING, BUT NOT
@c  LIMITED TO, THE IMPLIED WARRANTIES OF MERCHANTABILITY AND FITNESS FOR
@c  A PARTICULAR PURPOSE ARE DISCLAIMED. IN NO EVENT SHALL THE COPYRIGHT
@c  HOLDER OR CONTRIBUTORS BE LIABLE FOR ANY DIRECT, INDIRECT, INCIDENTAL,
@c  SPECIAL, EXEMPLARY, OR CONSEQUENTIAL DAMAGES (INCLUDING, BUT NOT
@c  LIMITED TO, PROCUREMENT OF SUBSTITUTE GOODS OR SERVICES; LOSS OF USE,
@c  DATA, OR PROFITS; OR BUSINESS INTERRUPTION) HOWEVER CAUSED AND ON ANY
@c  THEORY OF LIABILITY, WHETHER IN CONTRACT, STRICT LIABILITY, OR TORT
@c  (INCLUDING NEGLIGENCE OR OTHERWISE) ARISING IN ANY WAY OUT OF THE USE
@c  OF THIS SOFTWARE, EVEN IF ADVISED OF THE POSSIBILITY OF SUCH DAMAGE.


@node Developing

@html
@include style.css
@end html

@chapter Developing

This section details the development process with @emph{Kiss}.  The
details provided will be for IntelliJ but can be adapted to other
IDEs.  The use of an IDE is tremendously beneficial because of the
graphical debugging and intelligent code completion capabilities.

A typical development environment includes two separate web servers running
on two different ports (all running on your single development machine).  One serves the back-end REST services, and
the other serves the front-end HTML, JavaScript, and CSS files.  This
arrangement allows both front-end and back-end development without any
compiles, server reboots, file copies, or deployments.  Back-end and
front-end files can be edited and saved.  Their changes take effect
immediately.

Of course, in a production environment, only a single web server would be
utilized, and front-end and back-end changes take effect immediately
there too.

@section Overview

The following details the normal steps to boot up a development environment.
Once this environment is set up, it may remain active for the whole day.
There is rarely any reason to re-build or re-boot the development environment.

Note that all of the command-prompt commands are executed from the
root of your application.

@enumerate
@item
In a command-prompt, type:  @code{./view-log} (not needed on Windows)

This is where back-end messages appear. This command is not needed under Windows.
@item
In a second command-prompt, type:  @code{./bld develop} (@code{bld develop} on Windows)

This builds the system and runs the back-end and front-end servers.
@item
In the IDE, attach to the Java process at port 9000.

This will allow you to debug the back-end.
@item
From your browser, go to @uref{http://localhost:8000}

This is where you view and interact with your application.
@item
Open the @emph{Developer tools} from within the browser.

This is where you debug the front-end.
@item
Be sure to disable network caching on the browser. (Otherwise, changes you make to
the front-end will not immediately appear in the browser.)
@end enumerate

At this point, development, testing, and debugging can occur unabated.
There should be no need to rebuild or bring anything down.

Front-end changes will appear as soon as you re-load the page on your browser.
Back-end changes will take effect immediately.

@subsection IDE

While development with @emph{Kiss} does not require an IDE, most
developers utilize an IDE (Integrated Development Environment).  
@emph{Kiss} includes an effort to assist with IDE integration
(such as an @emph{ant} interface to our build system).

See the files under the @code{manual/IDE-Setup} directory for further
information.

@section Back-end Development

The back-end works differently in development and production
environments.  Although in both environments back-end and front-end
changes take effect immediately, the setup of the production environment
copies all files into the production environment, whereas in a
development environment the source and production code are split.  In
order to facilitate rapid and easy development, it is important that
source files be used rather than the production copies during the
development process.  

The system automatically detects the location of the application
source in most configurations.  However, this may be explicitly
set via a system environment variable (@code{KISS_ROOT}).  The
value of this environment variable should be the absolute path of the
root of the application source code.  The directory it indicates
should have a sub-directory named @code{src/main/backend}.



@subsection REST Server

There are two different methods of running the back-end REST server as
follows.

@subsubsection IDE

Running the IDE back-end requires the following:

@enumerate
@item
The IDE is completely configured.
@item
The system was built with the IDE.
@end enumerate

Typically, the IDE manages a Tomcat instance and serves up the core
back-end code serving the REST services.  That code detects that it is
running under an IDE and re-routes all back-end application files back
to the source directories.

Although back-end files are edited in source form and run in compiled
form, back-end code can be debugged (including breakpoints) as if
it were compiled before the system was booted.

Once the back-end server is up, application files changed under the
@code{src/main/backend} directory will take effect immediately.

@subsubsection BLD

Utilizing the included @emph{Kiss build system (bld)}, the many steps required 
to install and configure Tomcat and the IDE are unnecessary.  The whole process
(for the back-end portion) can be done as follows:

@enumerate
@item
From any stage (including having just downloaded the @emph{Kiss} system) type
the following:  @code{./bld develop}

Note:  Remember that all commands that start with @code{./} would drop
the @code{./} under Windows.
@end enumerate
At this point, the back-end will be running.  From within your IDE, you
can attach to the process at port 9000 to debug the back-end (including
breakpoints, etc.).

Remember, however, that you won't be able to use or debug the application until
the front-end server is started too.

@subsection Application Code

All communications between the front-end and back-end occur over REST
services you define.  Each REST service exists in its own
file or class.  Web methods are methods within those classes.

As architected, directories under @code{src/main/backend}
represent the application's REST services.  Each class/file under that
directory represents a web service.  The name of the class is the name
of the web service.

Instance methods within the web service class represent REST methods
for that web service.  Each web method is passed four arguments as follows:

@table @code
@item JSONObject injson
This represents the data that came from the front-end.
@item JSONObject outjson
This represents the data being returned to the front-end.  It is pre-initialized with an empty @code{JSONObject}.
@item Connection db
This is a pre-opened connection to the defined database (defined in @code{backend/application.ini})
@item MainServlet servlet
This is a rarely used servlet context argument.
@end table

Basically what happens is:

@enumerate
@item
The front-end makes a REST service call.
@item
The @code{Kiss} back-end receives the request.
@item
The user gets authenticated.
@item
A new database connection is formed, and a new database transaction is started.
@item
The requested web service is identified (and loaded and compiled if needed).
@item
The @code{outjson} object is filled in by the web service that is to be returned to the front-end.
@item
Upon completion of the REST service, @emph{Kiss} commits the transaction, closes the database connection, and returns @code{outjson} to the front-end.
@item
If, however, the REST service throws an exception, @emph{Kiss} rolls back the transaction, closes the database, and sends an error return to the front-end.
@end enumerate

Additional class and instance methods, that are not web methods, may
be defined and used within web service classes.

Of course during this process, @code{Kiss} handles many possible error conditions.

@anchor{cached_user_data}
@subsection Cached User Data

During the login process, user-specific data may be cached.  This
often occurs in the @code{login} Groovy method of the @code{Login}
application-specific class.

Every web service has access to this data via the following method:

@example
        servlet.getUserData().getUserData("dataName")
        servlet.getUserData().putUserData("dataName", "dataValue")
@end example

@xref{authentication,,Authentication}

@subsection Programming Languages

At this point, @emph{Kiss} supports the development of back-end REST
services in the Java, Groovy, or Common Lisp languages.  Groovy was
added first because it was easy, worked with the IDE well, and did all
that was needed.  (See the document @emph{GroovyOverview} included
with @emph{KISS}.) Java was added simply due to its natural
integration with the rest of the system.

With @emph{Kiss}, different web services can be written in different
languages.  You are not forced to use one or the other.

Common Lisp (ABCL) was added due to this author's love of that
language.  Unlike Groovy and Java, Lisp has an impedance mismatch with
the core @emph{Kiss} system that is written in Java.  For example, in
Java, one can have two methods with the same name in the same class
that differ only by their argument types.  This is not part of the
Common Lisp language.  Also, the foreign function interface in Lisp
requires some Lisp code to make the connection clear.  Due to this
connection code having to run, the Lisp interface is very slow on the
first call.  It is, however, reasonably fast on all subsequent calls.
The code that interfaces Java to Lisp is under the
@code{src/main/core/org/kissweb/lisp} directory.

Due to the easy and natural connection between Java and many other JVM
languages, interfaces to those languages are very easy.  It is
anticipated that support for many of these other languages (such as
Scala, Clojure, JRuby, Jython, and Kotlin) will likely follow,
especially if there are requests.

@subsection Cron

@emph{Kiss} has the ability to run any number of commands at specified
intervals.  For example, you could run a process every hour and
another process every Tuesday at 3 PM, etc.  This facility echoes 
the standard Unix or Linux @code{cron} facility.

The file that determines what gets run and when is @*
@code{src/main/backend/CronTasks/crontab}.  All tasks must be in
the @code{Groovy} language and exist in the
@code{src/main/backend/CronTasks} directory.

All contents of the @code{src/main/backend/CronTasks} can be changed
on a running system.  The change will be noticed and take effect.
There is no need to restart the system.

Look at the files in the @code{src/main/backend/CronTasks}
directory for samples and further documentation.

@subsection CORS

For security reasons, web servers prevent web services from
an application that doesn't come from the same origin.  This is
known as Cross-Origin Resource Sharing or CORS.  @emph{Kiss} fully
supports this security standard. 

@emph{Kiss} automatically enables full CORS protection in production
environments and allows CORS in the development environment.

@subsection Unit Tests

@emph{Kiss} supports @code{jUnit} unit tests.  The tests are located
under the @code{src/test} directory.  The directory hierarchy under
the @code{src/test} should mirror that of the @code{src/main}
directory.

Unit tests may be run as follows:

@enumerate
@item
Add or change unit tests or code to be tested.
@item
Build @emph{KissUnitTest.jar} by executing:
@example
    ./bld unit-tests
@end example
This creates the @code{work/KissUnitTest.jar} file which contains
the entire @emph{Kiss} system and unit tests.
@end enumerate

All tests can be run by executing the following command:
@example
java -jar work/KissUnitTest.jar  execute \
  --class-path work/KissUnitTest.jar  --scan-class-path
@end example
Alternatively, a single test can be run as follows:
@example
java -jar work/KissUnitTest.jar \
  execute \
  --class-path work/KissUnitTest.jar \
  --select-class org.kissweb.IniFileTest
@end example
where @code{org.kissweb.IniFileTest} is the full path to the desired test.

@anchor{Front-end Development}
@section Front-end Development

A separate server is used so that development files will be served
rather than the front-end that was present when the back-end server
was started.  This is the same whether you are using the IDE or
@emph{bld} back-end server modes.

Once the back-end and front-end servers are running, the front-end can
be debugged through your browser debugger (F12 on Chrome).

In order to make this work, there are two steps that
need to be followed.

@enumerate
@item
The front-end needs to know where the back-end is located.  This is
controlled in the file @code{src/main/frontend/index.js}.  That file
contains a line that looks as follows:
@example
    Server.setURL('http://localhost:8080');
@end example
As shipped, that setting should be good in most cases.  Adjust as needed.
@item
A server needs to be running to serve the front-end code.  The
@emph{Kiss} system comes with a simple server that performs this
function.  It is in a file named @code{SimpleWebServer.jar} on the
root of the @emph{Kiss} system.  From the root of the @emph{Kiss}
system, run the following command to run the front-end server:
@example
   ./serve
@end example
or on Windows:
@example
   serve.cmd
@end example
@end enumerate
(Source code to this simple server is available at 
@uref{https://github.com/blakemcbride/SimpleWebServer})

Once the front-end and back-end servers are up, you can access the development
environment through your browser at @code{http://localhost:8000}

If you disable the browser cache through the browser developer console
settings, changes you make to the front-end will appear by just
reloading the page.  On Chrome, for example, the browser cache can be
turned off by going into the page debugger (F12), then
@code{settings}, and then selecting @code{Disable cache (while DevTools
is open)}

You can now develop the front-end portion of your application by
editing files under the @code{src/main/frontend} directory.

@subsection Mobile Interface

It is often necessary to support mobile devices.  Although many
applications support mobile applications through pages that use
responsive design, it is often the case that the pages are so
different between various platforms that the use of whole new pages is
simpler and more effective -- especially for complex screens.

Although @emph{Kiss} has always supported responsive design,
@emph{Kiss} also supports the ability to have different pages for
different platforms.  If you look at @code{src/main/frontend/index.js}
you will see how @code{Kiss} handles this.  Basically, @code{Kiss}
detects the platform and loads different pages based on the platform.


@section Reports And Exports

Creating reports and exports requires both front-end and back-end components.
The back-end usually creates the file to be sent to the front-end.  It then returns
a path that the front-end can use to download the file.

@emph{Kiss} provides the infrastructure needed to support this facility.
@emph{Kiss} also manages and cleans up report files that are no longer needed.

In terms of producing reports, @emph{Kiss} leverages the facilities
provided by the common groff/tbl/mm utilities publicly available.
These facilities automatically handle paging, titles, page numbering,
tables, and overall formatting of your reports. See
@uref{https://www.gnu.org/software/groff}

In terms of producing CSV export files, see the back-end
@code{DelimitedFileWriter} class.

When files are produced by the back-end, they are sent to the
front-end by just providing the URL to the file.  At that point, the
front-end @code{Utils.showReport} takes the URL returned by the back-end
and downloads the file.

@anchor{authentication}
@section Authentication

@emph{Kiss} has built-in authentication.  However, each application has its
own method of storing and validating users.  Additionally, each application
may have its own user-specific data it may want to retain between web service
calls. @emph{Kiss} has a generic and easy way of handling these needs.

Application-specific user login and data are handled by the Groovy file
located in the @code{backend} directory named @code{Login.groovy}.
That file must have two methods: @code{login} and @code{checkLogin}.  
See that file for more details.

@xref{cached_user_data,,Cached User Data}

@section The Build System

As discussed previously, @emph{Kiss} does not use any of the popular
build systems such as @emph{Ant}, @emph{Maven}, or @emph{Gradle}.  The
reason for this is that traditional build systems employ a convention
over configuration philosophy.  This works extremely well if you're
building a very standard system.  You follow a set of conventions, and
the build system does the rest automatically.

The problem with this approach is that if you are trying to do
something new, the build system fights you every step of the way.  To
accomplish something unique, you either have to go through all kinds of
crazy and complex gyrations, or sometimes the build system just can't
do it.

@emph{Kiss} uses a new microservice architecture.  While the core of
@emph{Kiss} gets compiled ahead of time in a traditional way, the
application code gets compiled at runtime.  For this reason, when
building the system, the core needs to be compiled normally, but the
application code needs to be copied rather than compiled.  Getting
typical build systems to understand this turned out to be inordinately
complex.

The build system that comes with @emph{Kiss} (called @emph{bld}) is
extremely simple and makes it trivial and easy to do whatever you
like.  It is also smart and capable.  It can download and cache
@emph{jar} and other files from remote locations, and it only does what
needs to be done.

Additionally, it is easy to create an @emph{Ant} driver that just calls 
@emph{bld}.  This is done so that IDEs that have @emph{Ant} interfaces
can now work with @emph{bld}!

@emph{bld} is made up of only two Java files (and a few standard
@emph{jars}). One file is located at @code{src/main/core/org/kissweb/BuildUtils.java}.
This file is part of the core of @emph{Kiss} and shouldn't be changed.
The other file is located at @code{src/main/precompiled/Tasks.java}.
This file contains all of your application-specific build procedures.
In general, this file need not be touched unless your application requires
special build processes.  @code{Tasks.java} is never updated by the
@emph{Kiss} upgrade process, so it will have to be manually checked whenever an
update to @code{Kiss} is performed.

The methods (tasks) in @code{Tasks.java} call the methods in
@code{BuildUtils.java} to perform the build tasks such as copying
files, compiling Java files, creating jar files, etc.  The thing about
these methods is that you tell the system what to do.  Those utilities
first check to see if it has already been done, and if so, they don't
repeat it.  Thus, only what needs to be done occurs.

When upgrading the system (@xref{updates,Kiss Framework Updates}),
@code{Tasks.java} is never updated so that changes you make to that
file will not be wiped out.  It is up to you to reconcile the
differences between the updated @code{Tasks.java} file and what you
want done.

@section Using Artificial Intelligence

New AI/LLM systems have appeared (like @emph{Claude Code}) that can develop whole applications.
Fully embracing this new technology, @emph{Kiss} now includes files to help facilitate this mode of development.
In particular, @emph{Kiss} includes two files as follows:

@table @code
@item AI/KnowledgeBase.md
This file explains the @emph{Kiss} framework to an AI system.
@item AI/ApplicationDetails.md
This is a template you can use to describe the application you are building.
@end table

In order to use these, follow the following steps:

@enumerate
@item
Edit @code{ApplicationDetails.md} to describe the application you will be building.
@item Start up the AI system and tell it to read those two files.
@end enumerate

From that point, you can instruct the AI to build your application.

You may run @emph{Kiss} and test the application while the AI is developing your application.  If any errors appear, feed them to the AI
and it will fix them.


