@c -*-texinfo-*-

@c  Copyright (c) 2018 Blake McBride
@c  All rights reserved.
@c
@c  Redistribution and use in source and binary forms, with or without
@c  modification, are permitted provided that the following conditions are
@c  met:
@c
@c  1. Redistributions of source code must retain the above copyright
@c  notice, this list of conditions and the following disclaimer.
@c
@c  2. Redistributions in binary form must reproduce the above copyright
@c  notice, this list of conditions and the following disclaimer in the
@c  documentation and/or other materials provided with the distribution.
@c
@c  THIS SOFTWARE IS PROVIDED BY THE COPYRIGHT HOLDERS AND CONTRIBUTORS
@c  "AS IS" AND ANY EXPRESS OR IMPLIED WARRANTIES, INCLUDING, BUT NOT
@c  LIMITED TO, THE IMPLIED WARRANTIES OF MERCHANTABILITY AND FITNESS FOR
@c  A PARTICULAR PURPOSE ARE DISCLAIMED. IN NO EVENT SHALL THE COPYRIGHT
@c  HOLDER OR CONTRIBUTORS BE LIABLE FOR ANY DIRECT, INDIRECT, INCIDENTAL,
@c  SPECIAL, EXEMPLARY, OR CONSEQUENTIAL DAMAGES (INCLUDING, BUT NOT
@c  LIMITED TO, PROCUREMENT OF SUBSTITUTE GOODS OR SERVICES; LOSS OF USE,
@c  DATA, OR PROFITS; OR BUSINESS INTERRUPTION) HOWEVER CAUSED AND ON ANY
@c  THEORY OF LIABILITY, WHETHER IN CONTRACT, STRICT LIABILITY, OR TORT
@c  (INCLUDING NEGLIGENCE OR OTHERWISE) ARISING IN ANY WAY OUT OF THE USE
@c  OF THIS SOFTWARE, EVEN IF ADVISED OF THE POSSIBILITY OF SUCH DAMAGE.


@chapter Developing

This section details the development process with @emph{Kiss}.  The
details provided will be for intelliJ but can be adapted to other
IDE's.  The use of an IDE is tremendously beneficial because of the
graphical debugging and intelligent code completion capabilities.

A good development environment includes two separate servers running
on two different ports.  One serves the back-end REST services, and
the other serves the front-end HTML, JavaScript, and CSS files.  This
arrangement allows both front-end and back-end development without any
compiles, server reboots, file copies, or deployments.  Back-end and
front-end files can be edited and saved.  Their changes take effect
immediately.

Of course, in a production environment, only a single server would be
utilized.

@section Back-end Development

The back-end works differently in development and production
environments.  Although in both environments back-end and front-end
changes take effect immediately, setup of the production environment
copies all files into the production environment, whereas in a
development environment the souce and production code are split.  In
order to facilitate rapid and clear development, it is important that
souce files be used rather than the production copies during the
development process.

@subsection REST Server

Typically, the IDE manages a tomcat instance and serves up the core
back-end code serving the REST services.  That code detects that it is
running under an IDE and re-routes all back-end application files back
to the source directories.

Although back-end files are edited in source form and run in compiled
form, back-end code can be debugged (including break-points) as if
they were compiled before the system was booted.

Once the back-end server is up, application files changed under the
@code{src/main/application} directory will take effect immediately.

The back-end server should be started before the front-end server.

@subsection Application Code

All communications between the front-end and back-end occur over REST
services you define.  Each REST service method exists in its own, single
file.

As architected, each directory under @code{src/main/application}
represents a single REST service.  Each class / file under that
directory represents a web method.  The name of the class is the name
of the web method.

Each class must have a static main method.  This is the method that
the @emph{Kiss} core calls.  This method is passed four argument as
follows:

@table @code
@item JSONObject injson
This represents the data that came from the front-end.
@item JSONObject outjson
This represents the data being returned to the front-end.  It is pre-initialized with an empty @code{JSONObject}.
@item Connection db
This is a pre-opened connection to the defined database (defined in @code{application/KissInit.groovy})
@item MainServlet servlet
This is a rarely used servlet context argument.
@end table

Basically what happens is:

@enumerate
@item
The front-end make a REST service call.
@item
The @code{Kiss} back-end receives the request.
@item
The user gets authenticated.
@item
A new database connection is formed, and a new database transaction is started.
@item
The requested web service is identified (and loaded and compiled if needed).
@item
The @code{outjson} object that is filled in by the web service that is to be returned to the front-end.
@item
Upon completion of the REST service, @emph{Kiss} commits the transaction, closes the databse connection, and returns @code{outjson} to the front-end.
@item
If, however, the REST service threw an exception, @emph{Kiss} rolls back the transaction, closes the database, and sends an error return to the front-end.
@end enumerate

Of course during this process, @code{Kiss} handles many possible error conditions.

@subsection Programming Languages

At this point, @emph{Kiss} supports the develpment of back-end REST
services in the Java, Groovy, or Common Lisp languages.  Groovy was
added first because it was easy, worked with the IDE well, and did all
that was needed.  Java was added simply due to its natural integration
with the rest of the system.

With @emph{Kiss}, different web services can be written in different
languages.  You are not forced to use one or the other.

Common Lisp (ABCL) was added due to this author's love of that
language.  Unlike Groovy and Java, Lisp has an impedence mismatch with
the core @emph{Kiss} system that is written in Java.  For example, in
Java, one can have two methods with the same name in the same class
that differ only by their argument types.  This is not part of the
Common Lisp language.  Also, the foreign function interface in Lisp
requires some Lisp code to make the connection clear.  Due to this
connection code having to run, the Lisp interface is very slow on the
first call.  It is, however, reasonably fast on all subsequent calls.
The code that interfaces Java to Lisp is under the
@code{src/main/java/org/kissweb/lisp} directory.

Do to the easy and natural connection between Java and many other JVM
languages, interfaces to those languages is very easy.  It is
anticipated that support for many of these other languages (such as
Scala, Clojure, JRuby, Jython, and Kotlin) will likely follow,
especially if there are requests.

@anchor{Front-end Development}@section Front-end Development

The front-end can be debugged through your browser debugger (F12 on Chrome).

A separate server is used so that development files will be served
rather than the front-end that was present when the back-end server
was started.  In order to make this work, there are two steps that
need to be followed.

@enumerate
@item
The front-end needs to know where the back-end is located.  This is controlled in the file @code{src/main/webapp/index.js}.
That file contains a line that looks as follows:
@example
    Server.setURL('http://localhost:8080');
@end example
As shipped, that setting should be good in most cases.  Adjust as needed.
@item
A server needs to be running to serve the front-end code.  The
@emph{Kiss} system comes with a simple server that performs this
function.  It is in a file named @code{SimpleWebServer.jar} on the
root of the @emph{Kiss} system.  From the root of the @emph{Kiss}
system, run the following command to run the front-end server:
@example
   ./serve
@end example
or on Windows:
@example
   serve.cmd
@end example
@end enumerate

The back-end server should be started before the front-end server.

Once the front-end server is up, you can access the development
environment through your browser at @code{http://localhost:8000}

If you disable the browser cache, changes you make to the front-end
will appear by just reloading the page.  On Chrome, for example, the
browser cache can be turned off by going into the page debugger (F12),
then @code{settings}, and then select @code{Disable cache (while
DevTools is open)}

You can now develop the front-end portion of your application by
editing files under the @code{src/main/webapp} directory.




