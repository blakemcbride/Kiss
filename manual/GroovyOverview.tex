\input texinfo    @c -*-texinfo-*-
@c %**start of header
@setfilename GroovyOverview.html
@settitle Groovy Overview
@c %**end of header
@setchapternewpage odd
@codequoteundirected on
@codequotebacktick on
@tex
\global\def\linkcolor{0 0 1}  % blue
\global\def\urlcolor{0 0 1}   % blue
@end tex

@ifinfo
Copyright  @copyright{} 2023 Blake McBride
All rights reserved.
@sp 1
Redistribution and use in source and binary forms, with or without
modification, are permitted provided that the following conditions are
met:
@sp 1
1. Redistributions of source code must retain the above copyright
notice, this list of conditions and the following disclaimer.
@sp 1
2. Redistributions in binary form must reproduce the above copyright
notice, this list of conditions and the following disclaimer in the
documentation and/or other materials provided with the distribution.
@sp 1
THIS SOFTWARE IS PROVIDED BY THE COPYRIGHT HOLDERS AND CONTRIBUTORS
``AS IS'' AND ANY EXPRESS OR IMPLIED WARRANTIES, INCLUDING, BUT NOT
LIMITED TO, THE IMPLIED WARRANTIES OF MERCHANTABILITY AND FITNESS FOR
A PARTICULAR PURPOSE ARE DISCLAIMED. IN NO EVENT SHALL THE COPYRIGHT
HOLDER OR CONTRIBUTORS BE LIABLE FOR ANY DIRECT, INDIRECT, INCIDENTAL,
SPECIAL, EXEMPLARY, OR CONSEQUENTIAL DAMAGES (INCLUDING, BUT NOT
LIMITED TO, PROCUREMENT OF SUBSTITUTE GOODS OR SERVICES; LOSS OF USE,
DATA, OR PROFITS; OR BUSINESS INTERRUPTION) HOWEVER CAUSED AND ON ANY
THEORY OF LIABILITY, WHETHER IN CONTRACT, STRICT LIABILITY, OR TORT
(INCLUDING NEGLIGENCE OR OTHERWISE) ARISING IN ANY WAY OUT OF THE USE
OF THIS SOFTWARE, EVEN IF ADVISED OF THE POSSIBILITY OF SUCH DAMAGE.
@end ifinfo

@titlepage
@title Groovy Overview
@subtitle Part of the Kiss Web Development Framework
@c @subtitle @today{}
@subtitle April 4, 2023
@author by Blake McBride
@page
@vskip 0pt plus 1filll
Copyright  @copyright{} 2023 Blake McBride
All rights reserved.
@sp 1
Redistribution and use in source and binary forms, with or without
modification, are permitted provided that the following conditions are
met:
@sp 1
1. Redistributions of source code must retain the above copyright
notice, this list of conditions and the following disclaimer.
@sp 1
2. Redistributions in binary form must reproduce the above copyright
notice, this list of conditions and the following disclaimer in the
documentation and/or other materials provided with the distribution.
@sp 1
THIS SOFTWARE IS PROVIDED BY THE COPYRIGHT HOLDERS AND CONTRIBUTORS
``AS IS'' AND ANY EXPRESS OR IMPLIED WARRANTIES, INCLUDING, BUT NOT
LIMITED TO, THE IMPLIED WARRANTIES OF MERCHANTABILITY AND FITNESS FOR
A PARTICULAR PURPOSE ARE DISCLAIMED. IN NO EVENT SHALL THE COPYRIGHT
HOLDER OR CONTRIBUTORS BE LIABLE FOR ANY DIRECT, INDIRECT, INCIDENTAL,
SPECIAL, EXEMPLARY, OR CONSEQUENTIAL DAMAGES (INCLUDING, BUT NOT
LIMITED TO, PROCUREMENT OF SUBSTITUTE GOODS OR SERVICES; LOSS OF USE,
DATA, OR PROFITS; OR BUSINESS INTERRUPTION) HOWEVER CAUSED AND ON ANY
THEORY OF LIABILITY, WHETHER IN CONTRACT, STRICT LIABILITY, OR TORT
(INCLUDING NEGLIGENCE OR OTHERWISE) ARISING IN ANY WAY OUT OF THE USE
OF THIS SOFTWARE, EVEN IF ADVISED OF THE POSSIBILITY OF SUCH DAMAGE.
@sp 1
Windows and Microsoft are registered trademarks of
Microsoft Corporation.  @TeX{}@ is a trademark of the American
Mathematical Society. Other brand and product names are trademarks or
registered trademarks of their respective holders.
@sp 1
This manual was typeset with the @TeX{}@ typesetting system developed by
Donald Knuth utilizing the Texinfo format.
@end titlepage
@iftex
@c @parskip = .5@baselineskip
@font@smallfnt = cmr5
@font@crfont = cmr7
@end iftex

@c @summarycontents
@contents

@chapter Why Groovy?
Groovy is used for several reasons:

@enumerate
@item
Groovy is only a little different from Java.  For the most part, if
you know Java, you know Groovy.
@item
Groovy has several language features that make it significantly
convenient in many situations.  (Although it also has a few
annoyances too that you need to be mindful of!)
@item
Groovy normally compiles to regular Java class files, runs at full
speed, and interacts Java <-> Groovy is if all code was Java or
Groovy.
@item
Although Groovy can be compiled ahead-of-time just like Java, the
way I am using it is to compile it at runtime.  This makes loading
it take longer (although certainly sub-second) but once loaded, it
runs at the same speed as compiled Java code.  Doing it this way
has the following advantages:
@enumerate a
@item
I can modify the code while the system is running!  The way I
load files is to check the file date/time.  If it changes, I
re-compile and load it.
@item
It is always clear to me what is dynamic (can be changed on a
running system) and what is static (compiled ahead of time).
All Java files are compiled ahead-of-time.  All Groovy files
are compiled when loaded at runtime.  Note that this is just a
convention I use and not a limitation of either Java or Groovy.
@item
Interestingly, and for reasons unclear to me, dynamically
compiling and loading Groovy seems to be very significantly
faster than doing the same thing in Java.
@end enumerate
@end enumerate

@chapter Groovy -- The Good
@enumerate
@item
Groovy integrates with Java seamlessly.  Groovy can call Java just
as if everything was written in Groovy.  Java can call Groovy just
as if it were Java.  Java and Groovy can be mixed and matched
without a thought.
   
Although this is true of code compiled normally (ahead of time),
there are special steps required to call code loaded at runtime
from other code.  This is because the system has to find the code,
compile, and load it if necessary before execution.
@item
Groovy has several types of strings.  One of the best features of
Groovy are strings constants that are delineated with triple quotes
(""").  These strings support embedded newlines.  This means you
can have multi-line strings.  This comes in especially handy when
representing SQL.  For example:
@example
       String sql = """select *
                       from mytable
                       where field = 'abc'
                       order by some_field"""
@end example
@item
In Java, to compare two strings you have to use:
@example
      str1 != null && str1.equals(str2)
@end example
   However, in Groovy, you can do:
@example
      str1 == str2
@end example
In Groovy, if you wish to see if they are the same object, use:
@example   
      str1.is(str2)
@end example
@item
In Groovy there are some convenient mappings.  For example:
@example
   System.out.println("hello") becomes println("hello")
@end example
@item
The last statement in a method has an implicit "return".  Therefore,
you can write a method as follows:
@example   
   int add(int a, int b) @{
       a + b
   @}
@end example
@item
Lambdas and closures.  In most situations lambdas and closures in
Java are extremely clunky to use.  The net effect is that they are
rarely used.  On the other hand, lambdas and closures in JavaScript
are easy to define and use.  The net effect is that lambdas and
closures are used very, very frequently in JavaScript.
   
Groovy lambdas and closures are like JavaScript.  They're easy to
define and easy to use.  For example:
@example   
       static void meth1() @{
            def fun = @{ a, b ->
                double x = a + b
                x * 5
            @}
            meth1(fun)
        @}

        static meth1(Closure fun) @{
            println(fun(17.2, 4))
        @}
@end example

The same thing in Java is:

@example
        interface MyInterface @{
            double add(double a, double b);
        @}

        static void main(String[] args) @{
            AtomicInteger y = new AtomicInteger(4);
            MyInterface fun = (a, b) -> @{
                double x = a + b;
                y.set(5);
                return x * 5;
            @};
            meth1(fun);
            System.out.println(y.get());
        @}

        static void meth1(MyInterface fun) @{
            System.out.println(fun.add(17.2, 4));
        @}
@end example
    Note that Java:
@enumerate a
@item
requires an interface (Java has several pre-defined
interfaces that may be used)
@item	   
cannot directly change closure variables (y)
@item	
when calling the lambda, the lambda method name must be
explicitly named
@end enumerate
@item
In Groovy, the following are all treated like "false":
@itemize @bullet
@item
false
@item
0
@item
""
@item
[]
@item
null
@end itemize
@example
     Java:  if (str != null && !str.isEmpty()) @{ ... @}
     Groovy:  if (str) @{ ... @}
@end example
@item
Iteration conveniences
@table @code
@item      for (i in 1..10) ...
@item      1.upto(10) ...           
`it' is 1 to 10
@item      10.times ...             
loops from 0-9
@item      0.step(10,2) ...         
loops 0-0 in increments of 2
@item      myLst.each @{ ... @}       
loops each element. `it' has individual elements
@end table
@item
String interpolation.  $VAR puts the value of VAR into a string.
@example
     Example:  "the value of x is $x"
@end example
Strings with single quotes do not support string interpolation.
@example
     Example:  'this is a string'
@end example
@item
?. references the left side only if it is not null.  Therefore:
@example
    Java:  if (x != null && x.xyz()) @{ ... @}
    Groovy:  if (x?.xyz()) @{ ... @}
@end example
@item
Declaring an instance variable without a scope causes Groovy to
automatically make it a bean.
@item
In Groovy, you can declare generic types with `def'. e.g.
@example
        def x = "hello"  // x is a String
@end example
This is the same as the Java `var' keyword.
In Groovy, either can be used.
(It's also the same as just declaring:
Object x = "hello")
@item
An ArrayList can be defined with [] for example
@example
     def a = [1, 2, 3] // an ArrayList that can hold any type
     a[3] = 'Hello'    // extends array
@end example
@item
Groovy has many more significant conveniences
@end enumerate

@chapter Groovy -- The Bad
Unfortunately, Groovy does have, what this author believes to be, some
bad features.  In the end, however, it is believed that the good
features outweigh the bad ones.  The following list enumerates them.
@enumerate
@item
Java, C, and many other languages use the semicolon to specify the
end of a statement.  However, in Groovy, it is optional.  This
sometimes makes code difficult for a human to read.  Now you have
to view a larger context in order to mentally determine the end of
a statement.

In spite of this fact, I tend not to terminate statements with a
semicolon in Groovy because the IDE is constantly and annoyingly
reminding me that the semicolon is not needed.  I don't use them to
shut the IDE up.
@item   
Likewise, in some situations, Groovy allows the calling of
functions without parenthesis.  For example, instead of writing:
@example   
      mymeth(arg1, arg2);
@end example
   In groovy you can use:
@example
      mymeth arg1, arg2
@end example
Just as in the previous complaint, this increases the difficulty
for humans to evaluate the code.
@item
Groovy doesn't have the @code{char} type.  This means @code{'x'} is a string of
length 1.  However, you often want a character.  In Groovy, you
have to use:
@example   
      (char) 'x'
        or
      'x' as char
@end example
@item
Double constants must end in ``D'' or ``d'' otherwise they're not a double,
they're a @code{BigDecimal}!
@example
    3.14  // BigDecimal
    3.14d // double
@end example
This and the prior item have caused countless problems due to this unexpected
behavior.  Beware!
@item
Groovy does not make `private' elements private.  This is something
many have complained about, but they refuse to correct.  It is major
because when trying to understand a large system, one need look no
further than a single file for truly private elements.  If there is no
private, one must look throughout the entire system to determine of a
given element is used.  This is a significant issue!
@end enumerate    

@chapter Calling Groovy in Kiss

Groovy, like Java, may call compiled Java and Groovy code in the
normal way.  This is always the case.  However, in @emph{KISS}, Groovy
code is not compiled before the system is started up, as is normally
the case, but is compiled at runtime when used.  This is to support
the fact that all Groovy code in @emph{KISS} are microservices.
In @emph{KISS}, each Groovy class is an independent microservice.
This effects how Groovy methods are called in @emph{KISS}.

Groovy microservices in @emph{KISS}, may make three types of calls as follows:
@enumerate
@item
Intra-method Calls -- calls from one method to another in the same class
@item
Compiled Library Calls -- these are calls from Groovy to compiled Java
code (@emph{KISS} core code, Java or Groovy compiled libraries, etc)
@item 
Microservice Calls - these are calls from one Groovy method to a Groovy
method in another class / microservice
@end enumerate
In @emph{KISS}, the first two types of calls may be performed as you would normally
perform them.  There are no special considerations.

However type 3 calls require a special procedure.  This is to give @emph{KISS}
an opportunity to compile an load the microservice before it is executed.
See the manual for more details.

    

@bye

