@c -*-texinfo-*-

@c  Copyright (c) 2018 Blake McBride
@c  All rights reserved.
@c
@c  Redistribution and use in source and binary forms, with or without
@c  modification, are permitted provided that the following conditions are
@c  met:
@c
@c  1. Redistributions of source code must retain the above copyright
@c  notice, this list of conditions and the following disclaimer.
@c
@c  2. Redistributions in binary form must reproduce the above copyright
@c  notice, this list of conditions and the following disclaimer in the
@c  documentation and/or other materials provided with the distribution.
@c
@c  THIS SOFTWARE IS PROVIDED BY THE COPYRIGHT HOLDERS AND CONTRIBUTORS
@c  "AS IS" AND ANY EXPRESS OR IMPLIED WARRANTIES, INCLUDING, BUT NOT
@c  LIMITED TO, THE IMPLIED WARRANTIES OF MERCHANTABILITY AND FITNESS FOR
@c  A PARTICULAR PURPOSE ARE DISCLAIMED. IN NO EVENT SHALL THE COPYRIGHT
@c  HOLDER OR CONTRIBUTORS BE LIABLE FOR ANY DIRECT, INDIRECT, INCIDENTAL,
@c  SPECIAL, EXEMPLARY, OR CONSEQUENTIAL DAMAGES (INCLUDING, BUT NOT
@c  LIMITED TO, PROCUREMENT OF SUBSTITUTE GOODS OR SERVICES; LOSS OF USE,
@c  DATA, OR PROFITS; OR BUSINESS INTERRUPTION) HOWEVER CAUSED AND ON ANY
@c  THEORY OF LIABILITY, WHETHER IN CONTRACT, STRICT LIABILITY, OR TORT
@c  (INCLUDING NEGLIGENCE OR OTHERWISE) ARISING IN ANY WAY OUT OF THE USE
@c  OF THIS SOFTWARE, EVEN IF ADVISED OF THE POSSIBILITY OF SUCH DAMAGE.

@node Split System

@html
@include style.css
@end html

@chapter Split System

@emph{Kiss} comes as a complete system that includes both the front-end and back-end.
This works well for most situations.  However, sometimes a web application may have a
single back-end that serves several different front-ends.  Or, perhaps you prefer to keep
the front-end and back-end projects separate.  @emph{Kiss} has a mechanism to support the
ability to split the front-end and back-end.

This chapter will detail the steps needed to accomplish this.

@section Back-end-only System

A back-end-only system is a complete system minus the front-end portion of the system.
The git repo is also deleted since it is meaningless in this scenario.

@subsection Creating

In order to create a back-end-only system you start with a complete
@emph{Kiss} system cleanly checked out and delete the front-end
portion.  That leads you with a back-end-only system.  However, before
doing that, you must install the third-party libraries.  The following
commands will accomplish this whole process.

@example
./bld libs                   (Linux or Mac)
./remove-frontend remove
    or
bld libs                     (Windows)
remove-frontend remove
@end example

Once these command have been run, what you have left is a back-end-only system.  The git 
repository it came in will have been deleted as well as the front-end portion of the system.
The procedure for developing, deploying, and upgrading the system will be slightly modified as 
described herein.


@subsection Developing

The main difference between developing on a back-end-only system vs.\ a 
whole system is just that the command used to start the back-end is
slightly different.  The locations of all of the back-end pieces are
the same.  Once running, the system may be modified while running just
the same.

Use the following command to build and run the back-end:
@example
./bld developBackend         (Linux or Mac)
     or
bld developBackend           (Windows)
@end example

Viewing the server log is done in the same way as before.


@subsection Deploying

Deploying the back-ed is done just as before except now, instead of a single file containing the whole system, 
you have a single file that represents the back-end only.  When deploying the front-end and back-end portions,
they will be treated as two separate systems to the server.

@subsection Upgrading

Upgrading a back-end-only project is done the same way as upgrading a full system except that after the upgrade, you'll need to run the 
@code{remove-frontend} script again. @xref{updates,,System Updates}

@section Front-end-only System

A front-end-only system is a complete system minus the backend-end portion of the system.
The git repo is also deleted since it is meaningless in this scenario.

@subsection Creating

In order to create a front-end-only system you start with a complete
@emph{Kiss} system cleanly checked out and delete the back-end
portion.  That leads you with a front-end-only system.  However, before
doing that, you must install the third-party libraries.  The following
commands will accomplish this whole process.

@example
./bld libs                   (Linux or Mac)
./remove-backend remove
    or
bld libs                     (Windows)
remove-backend remove
@end example

Once these command have been run, what you have left is a front-end-only system.  The git 
repository it came in will have been deleted as well as the backend-end portion of the system.
The procedure for developing, deploying, and upgrading the system will be slightly modified as 
described herein.


@subsection Developing

Before running the front-end, you should start the back-end as described above.

There are no steps required to edit front-end files.  However, you must serve them.  This can be accomplished with the following command:

@example
./serve          (Linux or Mac)
    or
serve            (Windows)
@end example

Presuming the back-end is running, you can access the running system at @code{http://localhost:8000}

As before, you can edit front-end files while the system is running.

@subsection Deploying

The front-end must be distributed as a seperate @emph{war} file named @emph{frontend.war}.  
This file can be created with the following command:

@example
./make-frontend         (Linux or Mac)
     or
make-frontend           (Windows)
@end example

From your server's perspective, the front-end and back-end are two different systems.

@subsection Upgrading

Upgrading a front-end-only project is done the same way as upgrading a full system except that after the upgrade, you'll need to run the 
@code{remove-backend} script again. @xref{updates,,System Updates}
