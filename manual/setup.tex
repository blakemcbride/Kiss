@c -*-texinfo-*-

@c  Copyright (c) 2018 Blake McBride
@c  All rights reserved.
@c
@c  Redistribution and use in source and binary forms, with or without
@c  modification, are permitted provided that the following conditions are
@c  met:
@c
@c  1. Redistributions of source code must retain the above copyright
@c  notice, this list of conditions and the following disclaimer.
@c
@c  2. Redistributions in binary form must reproduce the above copyright
@c  notice, this list of conditions and the following disclaimer in the
@c  documentation and/or other materials provided with the distribution.
@c
@c  THIS SOFTWARE IS PROVIDED BY THE COPYRIGHT HOLDERS AND CONTRIBUTORS
@c  "AS IS" AND ANY EXPRESS OR IMPLIED WARRANTIES, INCLUDING, BUT NOT
@c  LIMITED TO, THE IMPLIED WARRANTIES OF MERCHANTABILITY AND FITNESS FOR
@c  A PARTICULAR PURPOSE ARE DISCLAIMED. IN NO EVENT SHALL THE COPYRIGHT
@c  HOLDER OR CONTRIBUTORS BE LIABLE FOR ANY DIRECT, INDIRECT, INCIDENTAL,
@c  SPECIAL, EXEMPLARY, OR CONSEQUENTIAL DAMAGES (INCLUDING, BUT NOT
@c  LIMITED TO, PROCUREMENT OF SUBSTITUTE GOODS OR SERVICES; LOSS OF USE,
@c  DATA, OR PROFITS; OR BUSINESS INTERRUPTION) HOWEVER CAUSED AND ON ANY
@c  THEORY OF LIABILITY, WHETHER IN CONTRACT, STRICT LIABILITY, OR TORT
@c  (INCLUDING NEGLIGENCE OR OTHERWISE) ARISING IN ANY WAY OUT OF THE USE
@c  OF THIS SOFTWARE, EVEN IF ADVISED OF THE POSSIBILITY OF SUCH DAMAGE.


@chapter System Setup


@section Pre-requisites

You should download and install the following pre-requisites.

@enumerate
@item
Java JDK 8 from @uref{https://www.oracle.com/java/index.html}
@item
Gradle from @uref{https://gradle.org}
@item
An SQL database server (e.g.@ PostgreSQL, Microsoft SQL Server, MySQL, Oracle, SQLite)
@item
IDE (e.g.@ @uref{https://www.jetbrains.com/idea,IntelliJ}, Netbeans, eclipse)
@item
GIT source code control system
@end enumerate

The system was built and tested with JDK 8, @uref{https://www.postgresql.org,PostgreSQL}, @uref{https://www.jetbrains.com/idea,IntelliJ}, and 
@uref{http://tomcat.apache.org,tomcat}.  Other environments such as IIS, Glassfish, eclipse, should work fine but may require some configuration.

Install the above per their instructions.

In addition to the Gradle build files, Kiss comes with IntelliJ project files that may be used out-of-the-box.  Other IDE's can import the Gradle build files.

@section Download Kiss


Kiss is located at @uref{https://github.com/kiss-web/Kiss}

It can be downloaded via the following command:

@code{git clone https://github.com/kiss-web/Kiss.git}

@section Setup And Configuration


Given that Kiss is for business applications, it authenticates its
users.  In order for this to work, there must be a database of valid
users.  This information is persisted in an SQL database.  Therefore a
database will need to be created.  As shipped, the system comes
configured as follows:

@multitable {Database user password} {PostgreSQL} 

@item Database type
@tab PostgreSQL
@item Host
@tab localhost
@item Database
@tab kiss
@item Database user
@tab postgres
@item Database user password
@tab postgres

@end multitable

This can be configured in the file @code{src/main/application/KissInit.groovy}
Valid options for the Database type are as follows:

@itemize @bullet
@item
PostgreSQL
@item
Microsoft SQL Server
@item
MySQL
@item
Oracle
@item
SQLite
@end itemize

Support for other databases is easy to add.

The remaining parameters should be self-explanatory.  Use the format shown in the example.

Create the database (named ``kiss'' above or whatever name you
configured).  After that, you can run the script file @code{init.sql}
to create the table and some initial data.

The default username is @emph{kiss}, and the default password is @emph{password}

@section Building The System

The system may be built in several way.  It can be build from the
command line using @emph{Gradle}, or at can be build from within your
favorite IDE.  Using an IDE has the great advantage of a built-in
debugger.  This section will outline usage of some of these scenarios.

@subsection Gradle 

While not the best choice when in development mode, @emph{Gradle} is a
good choice when generating the JavaDocs and possible also when
generating the final @emph{WAR} file.  @emph{Gradle} may be used as follows:

@table @code
@item gradle javadoc
JavaDocs end up in the @code{build/docs/javadoc} directory.
@item gradle war
The WAR file ends up in the @code{build/libs} directory.  This file can be just placed in the webapps directory of tomcat to deploy.
@item gradle clean
Removes all the generated files.
@item gradle appRun
Runs a sample server serving up the application.  The system will be
available from your local browser at
@code{http://localhost:8080/Kiss}.  Use @code{^C} to exit.
@end table

@subsection IntelliJ Ultimate

IntelliJ is a good IDE.  The free version may be inadequate for
@emph{Kiss} however.  IntelliJ should be configured with support for
@emph{Gradle} and @code{tomcat}.  You may set it up as follows:

@enumerate
@item
Get a fresh clone
@item
If you just entered IntelliJ select ``Import Project'', or if already in a project select New / Project from existing source.
@item
Select the file @emph{build.gradle}
@item
Accept all defaults
@item
Go to dropdown at the top of intelliJ screen and select Edit Configurations
@item
Click the green + in the upper left corner
@item
Select @emph{Tomcat Server} / @emph{Local}
@item
Change the Name to Kiss
@item
Click the Deployment tab
@item
Click the Green + in the middle of the screen (not the upper left)
@item
Select @emph{Artifact} and then @emph{Kiss.war} (exploded)
@item
Click "OK"
@end enumerate

To run the app in debug mode just click the green bug at the top of the screen.


@section What Do I Do With It Now?

What you have at this point is the beginnings of your new application.
@emph{Kiss} is provided as a running and deployable system.  It is
expected that you would modify what's here to suit your application
needs.

Besides this narrative, you would need the JavaDocs located under the
@code{build/docs/javadoc} directory for API specific documentation.
