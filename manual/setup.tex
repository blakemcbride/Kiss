@c -*-texinfo-*-

@c  Copyright (c) 2018 Blake McBride
@c  All rights reserved.
@c
@c  Redistribution and use in source and binary forms, with or without
@c  modification, are permitted provided that the following conditions are
@c  met:
@c
@c  1. Redistributions of source code must retain the above copyright
@c  notice, this list of conditions and the following disclaimer.
@c
@c  2. Redistributions in binary form must reproduce the above copyright
@c  notice, this list of conditions and the following disclaimer in the
@c  documentation and/or other materials provided with the distribution.
@c
@c  THIS SOFTWARE IS PROVIDED BY THE COPYRIGHT HOLDERS AND CONTRIBUTORS
@c  "AS IS" AND ANY EXPRESS OR IMPLIED WARRANTIES, INCLUDING, BUT NOT
@c  LIMITED TO, THE IMPLIED WARRANTIES OF MERCHANTABILITY AND FITNESS FOR
@c  A PARTICULAR PURPOSE ARE DISCLAIMED. IN NO EVENT SHALL THE COPYRIGHT
@c  HOLDER OR CONTRIBUTORS BE LIABLE FOR ANY DIRECT, INDIRECT, INCIDENTAL,
@c  SPECIAL, EXEMPLARY, OR CONSEQUENTIAL DAMAGES (INCLUDING, BUT NOT
@c  LIMITED TO, PROCUREMENT OF SUBSTITUTE GOODS OR SERVICES; LOSS OF USE,
@c  DATA, OR PROFITS; OR BUSINESS INTERRUPTION) HOWEVER CAUSED AND ON ANY
@c  THEORY OF LIABILITY, WHETHER IN CONTRACT, STRICT LIABILITY, OR TORT
@c  (INCLUDING NEGLIGENCE OR OTHERWISE) ARISING IN ANY WAY OUT OF THE USE
@c  OF THIS SOFTWARE, EVEN IF ADVISED OF THE POSSIBILITY OF SUCH DAMAGE.


@chapter System Setup

@anchor{Important}@section Important

The @emph{Kiss} system comes with its own build system, so it doesn't
use build systems such as @emph{ant}, @emph{maven}, or @emph{gradle}.
It works under Linux, Mac, Windows, etc. This included build system
will make things such as downloading remote libraries, installing and
configuring a web server, building @emph{Kiss}, and running
@emph{Kiss} for development purposes easy, intelligent, and automatic.
This new build system can be used in conjunction with your favorite IDE.

The build system (called it @emph{bld}) is small and written in Java.
Its source code is included with the @emph{Kiss} source code.  So, a
Java compiler will be needed before anything will work.  At this
point, Java version 8 is primarily used.  Java 11 has also been
tested.  Later Java versions should work as well.

When building the system for the first time, @emph{BLD} will
automatically download and cache required libraries (jar files), install and
configure the development server (tomcat), build the system, and it
can even be used to launch the development-mode server.  After building
the system the first time, application development proceeds without
any need for re-compiles or re-builds.

The program that runs @emph{BLD} is ``@emph{bld}'' under Unix-like
systems and ``@emph{bld.cmd}'' under Windows.  In this manual,
``@emph{bld}'' will be shown as ``@emph{./bld}''.  The ``./'' is
required under Linux and Mac but not under Windows.  When running on
Windows, use ``bld'' rather than ``./bld'' and the same for all the
other commands shown.

@section Super-Quick-Start

This is the simplest and shortest path to a running system.  It assumes:

@enumerate
@item
you have a JDK installed
@item
you have an Internet connection
@end enumerate

After doing a @code{git clone}, all that is needed is the following:

@example
    ./bld  develop                      [Linux, Mac, BSD, etc.]
         -or-
    build-builder                       [Windows]
    bld  develop                        [Windows]
@end example

This will build the system, install tomcat, deploy the app, and run
the server.  At this point you will be able to go your browser at the
following URL:

@example
    http://localhost:8000
@end example

Be sure not to use port 8080.  Although port 8080 will appear to work,
you will not be able to do front-end development while the system is
running.  Port 8000 will allow front-end development while the system
is running.

At this point, you can do all development without any build
procedures.  You can add or change anything on the front-end or
back-end while the system is running.

All application back-end code is located under the @code{src/main/application}
directory.

All application front-end code is located under the @code{src/main/webapp}
directory.

If you change anything in either place, the system will notice the change 
and deliver it with the next request.


@anchor{Quick-Start Checklist}@section Quick-Start Checklist

This is a more detailed and expanded description of the same
super-quick-start.

The following enumerates the steps necessary to get the system up and running:

@enumerate
@item
@xref{Pre-requisites}
@item
@xref{Download Kiss}
@item
@xref{Setup And Configuration}
@item
The development server can be run without an IDE by running: @code{./bld develop}
@item
Once the server starts up, you can access it on your browser by going to 
@code{http://localhost:8000}  You can also debug the back-end by
attaching to the process at port 9000.

Alternatively, your IDE can be configured to run the development
process entirely through it.
@end enumerate

@noindent Once the development server is running under @emph{bld}, you can stop it 
by hitting any key.

@section Runtime Environments

As shipped, there are two different environments that @emph{Kiss} may
run in as follows:

@enumerate
@item
Development
@item 
Production
@end enumerate

The @emph{Production} scenario is created with a single command
(@code{./bld war}) and produces a single @code{war} file (located in
the @code{build.work} directory) that can be deployed to your production
environment.

Before doing anything with the @emph{Development} environment, it is
important that you have the @emph{JAVA_HOME} environment variable set
correctly to the root of your JDK (Java Developer Kit).  Doing this
varies according to the OS you are using and various other Java
installation possibilities.  There are plenty of instructions on the
Internet for this.

The @emph{Development} environment consists of two servers.  One
serves the back-end REST services, and the second serves the front-end
HTML, CSS, and JavaScript files.  By using this method, both front-end
and back-end source files can be changed on a running system and take
effect immediately without any builds, compiles, server reboots,
re-deploys, or file copies.  (This is also true of a production
environment -- with a single server -- when the new files are put in place.)

Back-end REST services are debugged, and edited through the IDE.
Saving a source file is all that is needed to have it take effect.

The front-end (HTML, CSS, and JavaScript files) are served by a simple
server supplied with the @emph{Kiss} system.  @xref{Front-end
Development}.  (Source code to this server is available at
@uref{https://github.com/blakemcbride/SimpleWebServer}) Debugging the
front-end is done through the browser debugger.  There is no setup,
and the front-end server runs by executing a single command.

@anchor{Pre-requisites}@section Pre-requisites

You should download and install the following pre-requisites.

@enumerate
@item
Java JDK 8 from many sources including @uref{https://www.azul.com/downloads/zulu-community} or 
@uref{https://docs.aws.amazon.com/corretto/latest/corretto-8-ug/downloads-list.html} or
@uref{https://developers.redhat.com/products/openjdk/download} or
@uref{https://adoptopenjdk.net}  (Java 11 has also been tested.)
@item
An SQL database server (e.g.@ PostgreSQL, Microsoft SQL Server, MySQL,
Oracle, SQLite)
@item
IDE (e.g.@ @uref{https://www.jetbrains.com/idea,IntelliJ}, NetBeans, eclipse)
@item
GIT source code control system
@end enumerate

Correctly setting the @emph{JAVA_HOME} environment variable to the root of your
JDK is necessary.  Setting this varies from OS to OS and also depends on where it
gets installed.  Instructions for setting this variable are all over the Internet.

The system was built and tested with JDK 8 and 11,
@uref{https://www.postgresql.org,PostgreSQL},
@uref{https://www.jetbrains.com/idea,IntelliJ}, and
@uref{http://tomcat.apache.org,tomcat}.  Other environments such as
IIS, Glassfish, eclipse, should work fine but may require some
configuration.

@emph{Kiss} comes with its own build system.  @xref{Important}

Install the above according to their instructions.

@anchor{Download Kiss}@section Download Kiss

Kiss is located at @uref{https://github.com/blakemcbride/Kiss}

It can be downloaded via the following command:

@code{git clone https://github.com/blakemcbride/Kiss.git}

@section Documentation

The @emph{Kiss} documentation consists of three parts; this manual,
the detailed back-end API documentation contained in the JavaDocs, and
the detailed front-end API documentation.  The JavaDocs do not come
with the system, but you can generate them yourself with what is
provided.  @xref{javadoc,,Creating JavaDocs}.

This manual may be created in two forms.  The first is in an HTML
form.  The system comes with this.  You can also generate a nicely
formatted PDF file with the following commands (if you have all of the
formatting tools installed):

@example
cd manual
make Kiss.pdf
@end example

@noindent
Updates to the HTML file are achieved with the following commands:
@example
cd manual
make
@end example

All of the documentation can be accessed with your browser.  For
example, if the root of @emph{Kiss} is located at
@code{/my/home/path/kiss} then you will be able to access the three
manuals at the following URL's:

@example
file:///my/home/path/kiss/manual/man/index.html

file:///my/home/path/kiss/build.work/javadoc/index.html

file:///my/home/path/kiss/manual/jsdoc/index.html
@end example

@anchor{Setup And Configuration}@section Setup And Configuration

The system is configured by the contents of a single file
@code{src/main/application/KissInit.groovy} A reboot of the web server
is required if any of the parameters in this file are changed.

Given that @emph{Kiss} is for business applications, it authenticates its
users.  In order for this to work, there is usually a database of valid
users.  This information is persisted in an SQL database.  Therefore a
database is normally required.  However, for testing purposes, if no
database is configured, the system will still run and allow any
username and password to succeed.

As shipped, the system comes configured as follows:

@multitable {Database user password} {PostgreSQL} 
@item Database type
@tab SQLite
@item Host
@tab localhost
@item Database
@tab DB.sqlite
@item Database user
@tab [empty]
@item Database user password
@tab [empty]
@end multitable

Valid options for the Database type are as follows:

@itemize @bullet
@item
PostgreSQL
@item
MicrosoftServer
@item
MySQL
@item
Oracle
@item
SQLite
@end itemize

Support for other databases is easy to add.

@emph{setMaxWorkerThreads} defines how many REST services may be
processed in parallel.  Service requests beyond this are placed in a
FIFO queue and processed as worker threads become available.  This
capability drastically improves the system's ability to handle a large
number of simultaneous users.

The remaining parameters should be self-explanatory.  Use the format
shown in the example.

Although @emph{Kiss} comes with a default demo database, another one
should be configured in live or more substantial development
environemnts.  An SQL script file, named @code{init.sql}, is included
with the system to initialize said database.  Application specific tables
may be added to this database.

The default username is @emph{kiss}, and the default password is
@emph{password}

@section Building The System

Although the system may be built with the included build system
(called @emph{bld}) @emph{or} your favorite IDE, the @emph{bld}
system should be used for the initial step which downloads the external
dependencies (jar files).

The build system included with @emph{Kiss} (called @emph{bld}) has
been tested on Linux, Mac, and Windows.  The system also includes an
@emph{Ant} build file (named @emph{build.xml}) that is only used for
IDE integration with the included @emph{bld} system.

The build system included with Kiss is written in Java and located
under the @emph{builder} directory.  This build system also includes
two driver batch files / shell scripts used to build and run the build
system.  All that is needed to use this system is a Java compiler.
(As a side note - this build system is generic and can be used to
build other types of projects.)

The build process is run from the command-line.  No IDE is necessary.
There is no specific IDE integration.  None is needed because the system
is rarely built.  After the first build, application development is done
without any build process.

The build system, which comes in source form, must be built before it
can be used to build @emph{Kiss}.  Under Linux, Mac, BSD, etc., the
build system gets built automatically.  However, under Windows, the 
following command is needed once to build the build system:

@example
    build-builder             [Windows]
@end example

Once the build system has been built, you can see what operations it
can perform by typing:

@example
    ./bld listTasks           [Linux, Mac, BSD, etc.]
        -or-
    bld listTasks             [Windows]
@end example

Those tasks that require prior tasks will evoke the dependent tasks
automatically.  The system is smart enough not to repeat tasks that
are unneeded.

The main tasks that will be of interest to you are as follows:

@table @code
@item libs
This task is only required if you intend to use your IDE to build the
remainder of the system.  It installs the required dependencies.
@item develop
This will cause the entire system to build (not repeating unnecessary
steps) and start up a tomcat instance to run the system in a
development mode.  The system will be available from your local
browser at @code{http://localhost:8000}.  You may debug the
application by attaching to the running @emph{tomcat} server at port
9000.
@item war
This will cause the system to generate the single file needed by a
production system.  It will end up in @emph{build.work/Kiss.war}
@end table

Other tasks which may be useful are as follows:

@table @code
@item clean
This task removes all files built but retains files that were downloaded from
repositories (although @emph{bld} caches those files anyway.)
@item realclean
This removes all built and downloaded files so the system should be everything
you need to build it without any extraneous files
@item all
This performs all of the steps necessary to setup and build the system but
doesn't start up the tomcat server.
@item javadoc
This task creates the javadoc files that end up in @code{build.work/javadoc}
@item kisscmd
This task creates a command-line JAR that can be used in a non-web, command-line application.
This is useful when creating applications that perform various utility functions.  This
JAR cannot be used in any web environment.  See @code{src/main/java/org/kissweb/Main.java}
@end table

@xref{Important} and @xref{Quick-Start Checklist}.

@subsection Using an IDE

Most IDE's can be used to develop and debug the application.  There are two
way to do this as follows:

@enumerate
@item
Using @emph{bld} to build and run the development environment.
@item
Using your IDE to build and run the development environment.
@end enumerate

Using @emph{bld} to build and run the development environment is the
easiest to start off with but is somewhat klunky on an ongoing basis.
Its main advantage is that it is portable and doesn't require a lot of
IDE configuration.  The back-end development server is configured and
started by @emph{bld} by simply running @emph{./bld develop} After
that, the IDE can be used to debug it by @emph{attaching} to port
9000.

Using your IDE to run the entire process is a bit tedious to initially
setup but makes the entire process simpler from that point forward.
Unfortunately, configuring your IDE is completely different for each
IDE.  Instructions for setting up the Ultimate (commercial) version of
IntelliJ and NetBeans are included as follows.  Other IDE's may be
configured according to the instructions that are typically associated
with the IDE.

Also, bear this in mind, the build process used by @emph{bld} and that
of the IDE are not the same.  Use one or the other.


@subsubsection IntelliJ (Ultimate)

Although the Ultimate (commercial) version of IntelliJ costs money, it
is probably the most powerful, reliable, and best supported IDE.
@emph{Kiss} works well with it.  On the other hand, like other IDE's,
IntelliJ can be a bit tedious to configure properly.  The following
steps can be used to configure IntelliJ.

@enumerate
@item
Download dependencies.  Even though IntelliJ will be used to build,
debug, and maintain the system, the @emph{Kiss} build system
(@emph{bld}) should initially be used to install the dependencies.
And even before that, the build system itself must be built.  All this
can be accomplished with the following commands:
@enumerate a
@item
build-builder                [under Windows only]
@item
./bld libs                   [Windows without the ./]
@end enumerate
@item
Install tomcat.  You must install a copy of the @emph{tomcat} server.
You can do this by either installing it in a typical way from the
Apache Tomcat site, or you can execute the following @emph{bld}
command to create a local tomcat installation for you:
@enumerate a
@item
./bld setupTomcat
@end enumerate
If @emph{bld} is used, it will setup tomcat in a directory named
@emph{tomcat} under the root of the @emph{Kiss} installation.
@item
Configure IDE for tomcat.  This can be done from within the IDE by:
@enumerate a
@item
Go to File / Settings / Build, Execution, Deployment / Application Servers.
@item
Click the + in the upper middle of the popup, and then select @emph{Tomcat Server}
@item
The home and base should both be set to the root of the tomcat directory.  This
will be an absolute rather than a relative path.
@end enumerate
@item
Import the project.  From within the IDE:
@enumerate a
@item
Select @emph{Import Project}
@item
Select the root directory of the @emph{Kiss} system.
@item
Select @emph{Create project from existing sources}
@item
Accept defaults but when you get to the @emph{Source files} page, de-select the 
@emph{builder} directory.
@item
On the libraries page, de-select @emph{commons} and @emph{SimpleWebServer}.
@item
Accept the remaining defaults until the project is created.
@end enumerate
@item
Configure artifact.  The artifact is the end files that IntelliJ deploys to tomcat.  
These are the files that typically end up in a WAR file used by tomcat.  Configuring
this is done with the following IntelliJ steps:
@enumerate a
@item
Right-click on the top node in the @emph{Projects} pane and select
@emph{Open Module Settings}.
@item
Select @emph{Artifacts}
@item
Click the + in the upper middle of the popup and then select @emph{Web
Application: Exploded} / @emph{From Module} and then select the only module.
@item
Check @emph{Include in project build}
@item
Expand @emph{Kiss} under @emph{Available Elements} and double-click on @emph{libs}
@item
Click @emph{OK}
@end enumerate
@item
Add a run configuration.  From within the IDE:
@enumerate a
@item
Select @emph{Run} / @emph{Edit configurations...}
@item
Click the + in the upper left corner of the popup and select @emph{Tomcat Server} / 
@emph{Local}.
@item
Change Name to @emph{Kiss}
@item
Verify the @emph{Application server} (use the tomcat we configured for this project).
The other defaults should be correct.
@item
Click the @emph{Deployment} tab
@item
Click the + on the right side of the popup and select @emph{Artifact...}
@item
Click on the @emph{Startup/Connection} tab
@item
Under the @emph{Debug} configuration and then @emph{again} under the
@emph{Run} configuration, add an environment variable named
@emph{KISS_ROOT} and set its value to the path of the root of your
@emph{Kiss} system.
@end enumerate
@end enumerate

After this configuration is performed, you can use the IDE's native
build and debug processes for development.  However, @emph{bld} should
still be used to produce the production WAR file.


@subsubsection NetBeans

NetBeans is currently under significant transition since it joined the Apache
organization.  NetBeans 11.2 did not work correctly with tomcat, however,
the nightly build as of February 28, 2020 did.  So, we expect that any version 
@emph{after} 11.2 should work.  This is how we configured the IDE.  In all cases,
when not specified otherwise. leave the defaults.

@enumerate
@item
Download dependencies.  Even though NetBeans will be used to build,
debug, and maintain the system, the @emph{Kiss} build system
(@emph{bld}) should initially be used to install the dependencies.
And even before that, the build system itself must be built.  All this
can be accomplished with the following commands:
@enumerate a
@item
./build-builder
@item
./bld libs
@end enumerate
@item
Install tomcat.  You must install a copy of the @emph{tomcat} server.
You can do this by either installing it in a typical way from the
Apache Tomcat site, or you can execute the following @emph{bld}
command to create a local tomcat installation for you:
@enumerate a
@item
./bld setupTomcat
@end enumerate
If @emph{bld} is used, it will setup tomcat in a directory named @emph{tomcat} under
the root of the @emph{Kiss} installation.
@item
Configure tomcat.  From within NetBeans, do the following:
@enumerate a
@item
Select @emph{Tools} and then @emph{Servers}
@item
Click @emph{Add Server...}, and then select @emph{Apache Tomcat or TomEE}
@item
Select the server location.  @emph{Catalina Base} should be blank.
@item
Set both the @emph{Username} and @emph{Password} to @emph{admin}, and then click 
@emph{Finish}
@end enumerate
@item
Import the Project through the IDE as follows:
@enumerate a
@item
Select @emph{File} and then @emph{New Project}
@item
Select @emph{Java Web} under @emph{Java with Ant}, and then select
@emph{Web Application with Existing Sources} on the right, and then
click @emph{Next}
@item
Under @emph{Location} select the root of the @emph{Kiss} system.  The
remaining defaults on that tab are fine.  Click @emph{Next}.
@item
On the @emph{Server and Settings} tab, be sure to select the
appropriate tomcat server.
@item
The following should be set on the @emph{Existing Sources and Libraries} page:
@enumerate A
@item
Set the @emph{Web Pages Folder} by selecting the @code{src/main/webapp}
directory.
@item
Set the @emph{WEB-INF Content} by selecting the @code{src/main/java} directory.
@item
Set the @emph{Libraries Folder} by selecting the @code{libs} folder
under the @emph{Kiss} root.
@item
Remove what is in the @emph{Sources Package Folders} section and add the two paths
@emph{src/main/application} and @emph{src/main/java}, and then click @emph{Finish}.
@end enumerate
@end enumerate
@item
Set the source folder names by:
@enumerate a
@item
Right-click on the root of the project in the @emph{Projects} pane and
select @emph{Properties}.
@item
On the @emph{Sources} page, you will see two labels in the
@emph{Source Package Folders} section that we need to change.  Change
the @emph{application} one to @emph{Application} and the second to @emph{Kiss Core}.
@item
Click @emph{OK}
@end enumerate
@end enumerate


After this configuration is performed, you can use the IDE's native
build and debug processes for development.  However, @emph{bld} should
still be used to produce the production WAR file.


@section What Do I Do With It Now?

What you have at this point is the beginnings of your new application.
@emph{Kiss} is provided as a running and deployable system.  It is
expected that you would modify what's there to suit your application
needs.

Besides this narrative, you would need the JavaDocs located under the
@code{build.work/javadoc} directory for back-end API specific documentation and
front-end API reference located in the @emph{manual/jsdoc} directory.
