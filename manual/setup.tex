@c -*-texinfo-*-

@c  Copyright (c) 2018 Blake McBride
@c  All rights reserved.
@c
@c  Redistribution and use in source and binary forms, with or without
@c  modification, are permitted provided that the following conditions are
@c  met:
@c
@c  1. Redistributions of source code must retain the above copyright
@c  notice, this list of conditions and the following disclaimer.
@c
@c  2. Redistributions in binary form must reproduce the above copyright
@c  notice, this list of conditions and the following disclaimer in the
@c  documentation and/or other materials provided with the distribution.
@c
@c  THIS SOFTWARE IS PROVIDED BY THE COPYRIGHT HOLDERS AND CONTRIBUTORS
@c  "AS IS" AND ANY EXPRESS OR IMPLIED WARRANTIES, INCLUDING, BUT NOT
@c  LIMITED TO, THE IMPLIED WARRANTIES OF MERCHANTABILITY AND FITNESS FOR
@c  A PARTICULAR PURPOSE ARE DISCLAIMED. IN NO EVENT SHALL THE COPYRIGHT
@c  HOLDER OR CONTRIBUTORS BE LIABLE FOR ANY DIRECT, INDIRECT, INCIDENTAL,
@c  SPECIAL, EXEMPLARY, OR CONSEQUENTIAL DAMAGES (INCLUDING, BUT NOT
@c  LIMITED TO, PROCUREMENT OF SUBSTITUTE GOODS OR SERVICES; LOSS OF USE,
@c  DATA, OR PROFITS; OR BUSINESS INTERRUPTION) HOWEVER CAUSED AND ON ANY
@c  THEORY OF LIABILITY, WHETHER IN CONTRACT, STRICT LIABILITY, OR TORT
@c  (INCLUDING NEGLIGENCE OR OTHERWISE) ARISING IN ANY WAY OUT OF THE USE
@c  OF THIS SOFTWARE, EVEN IF ADVISED OF THE POSSIBILITY OF SUCH DAMAGE.


@chapter System Setup

@anchor{Important}@section Important

The @emph{Kiss} system uses @emph{gradle} as its build system.  This
works under Linux, Mac, Windows, etc.  Although you need to install
Java before anything will work, you do not need to install
@emph{gradle}.  When you first try to build the system, it will detect
that you hadn't installed @emph{gradle} and it will download it and
install it for you.  This means it will take a bit longer on first
build.  @emph{Gradle} also downloads allof the required libraries the
first time too.  So again, the first build will be slow.  Subsequent
builds are fast, and since @emph{Kiss} uses microservices, most of the
time you don't even have to build at all.

The program that runs @emph{gradle} is ``@emph{gradlew}'' under Unix like
systems and ``@emph{gradlew.bat}'' under Windows.  In this manual,
``@emph{gradlew}'' will be shown as ``@emph{./gradlew}''.  The ``./'' is
required under Linux and Mac but not under Windows.  When running on
Windows, use ``gradlew'' rather than ``./gradlew''.


@anchor{Quick-start Checklist}@section Quick-start Checklist

The following enumerates the steps necessary to get the system up and running:

@enumerate
@item
@xref{Pre-requisites}
@item
@xref{Download Kiss}
@item
@xref{Setup And Configuration}
@item
Run the server by running @code{./gradlew tomcatRun}
@item
You can access the server by going to @code{http://localhost:8080}
@end enumerate

@noindent Use @code{^C} to stop the server.

@section Runtime Environments

As shipped, there are three different environments that @emph{Kiss} may
run in as follows:

@enumerate
@item
Test
@item
Development
@item 
Production
@end enumerate

The @emph{Test} environment is started with @code{./gradlew tomcatRun} but
is only good for a quick test to see if the system is running.  It is
neither good in a development nor production environment.

The @emph{Production} scenario is created with a single command 
(@code{./gradlew war}) and produces a single @code{war} file that
can be deployed.

Unfortunately, the @emph{Development} environment takes many steps to
setup.  This is largely due to the complexity and variations between
IDE's.  However, once the environment is setup, developing with
@emph{Kiss} is smooth and easy.

The @emph{Development} environment consists of two servers.  One serves
the back-end REST services, and the second serves the front-end HTML, CSS, 
and JavaScript files.  By using this method, both front-end and back-end
source files can be changed on a running system and take effect immediately
without any server reboots, re-deploys, or file copies.  (This is also
true of a production environment when the new files are put in place.)

Back-end REST services are served, debugged, and edited through the IDE.
Saving a source file is all that is needed to have it take effect.
Unfortunately, setting up IDE's to do this takes several steps and is different
for each IDE.

The front-end (HTML, CSS, and JavaScript files) are served by a simple
server supplied with the @emph{Kiss} system.  @xref{Front-end Development}.
Debugging is normally done through the browser debugger.
There is no setup, and it runs by executing a single command.

@anchor{Pre-requisites}@section Pre-requisites

You should download and install the following pre-requisites.

@enumerate
@item
Java JDK 8 from @uref{https://www.oracle.com/java/index.html}
@item
An SQL database server (e.g.@ PostgreSQL, Microsoft SQL Server, MySQL, Oracle, SQLite)
@item
IDE (e.g.@ @uref{https://www.jetbrains.com/idea,IntelliJ}, Netbeans, eclipse)
@item
GIT source code control system
@end enumerate

The system was built and tested with JDK 8, @uref{https://www.postgresql.org,PostgreSQL}, @uref{https://www.jetbrains.com/idea,IntelliJ}, and 
@uref{http://tomcat.apache.org,tomcat}.  Other environments such as IIS, Glassfish, eclipse, should work fine but may require some configuration.

@emph{Gradle} is the build system used.  The @emph{Kiss} system
automatically installs @emph{gradle} so you need not do anything
special.  @xref{Important}

Install the above per their instructions.

With the proper plugins, all of the IDE's can import the Gradle build files.

@anchor{Download Kiss}@section Download Kiss


Kiss is located at @uref{https://github.com/blakemcbride/Kiss}

It can be downloaded via the following command:

@code{git clone https://github.com/blakemcbride/Kiss.git}

@section Documentation

The @emph{Kiss} documentation consists of three parts; this manual, the detailed back-end API documentation contained in the JavaDocs,
and the detailed front-end API documentation.
The JavaDocs do not come with the system, but you can generate them yourself with what is provided.  @xref{javadoc,,Creating JavaDocs}.

This manual may be created in two forms.  The first is in an HTML
form.  The system comes with this.  You can also generate a nicely
formatted PDF file with the following commands (if you have all of the tools installed):

@example
cd manual
make Kiss.pdf
@end example

@noindent
Updated to the HTML file are achieved with the following commands:
@example
cd manual
make
@end example

All of the documentation can be accessed with your browser.  For example,
if the root of @emph{Kiss} is located at @code{/my/home/path/kiss} then you will be able to access the three
manuals at the following URL's:

@example
file:///my/home/path/kiss/manual/man/index.html

file:///my/home/path/kiss/build/docs/javadoc/index.html

file:///my/home/path/kiss/manual/jsdoc/index.html
@end example

@anchor{Setup And Configuration}@section Setup And Configuration

The system is configured by the contents of a single file @code{src/main/application/KissInit.groovy}
A reboot of the web server is required if any of the parameters in this file are changed.

Given that Kiss is for business applications, it authenticates its
users.  In order for this to work, there is usually a database of valid
users.  This information is persisted in an SQL database.  Therefore a
database is normally required.  However, for testing purposes, if no
database is configured, the system will still run and allow any
username and password to succeed.

As shipped, the system comes configured as follows:

@multitable {Database user password} {PostgreSQL} 
@item Database type
@tab PostgreSQL
@item Host
@tab localhost
@item Database
@tab not configured
@item Database user
@tab postgres
@item Database user password
@tab postgres
@end multitable

Valid options for the Database type are as follows:

@itemize @bullet
@item
PostgreSQL
@item
MicrosoftServer
@item
MySQL
@item
Oracle
@item
SQLite
@end itemize

Support for other databases is easy to add.

@emph{setMaxWorkerThreads} defines how many REST services may be processed in parallel.  Service requests
beyond this are placed in a FIFO queue and processed as worker threads become available.  This capability
drastically improves the system's ability to handle a large number of simultaneous users.

The remaining parameters should be self-explanatory.  Use the format
shown in the example.

Create the database (named ``kiss'' above or whatever name you
configured).  After that, you can run the script file @code{init.sql}
to create the table and some initial data.

The default username is @emph{kiss}, and the default password is
@emph{password}

@section Building The System

The system may be built in several way.  It can be build from the
command line using @emph{Gradle}, or it can be build from within your
favorite IDE.  Using an IDE has the great advantage of a built-in
debugger.  This section will outline usage of some of these scenarios.

@subsection Gradle 

@emph{Gradle} is a build system that describes and automates the build process.
@emph{Kiss} utilizes @emph{Gradle} to automate its build and debug processes.
Most IDE's are able to utilize the @emph{Gradle} build system.
@emph{Gradle} may be used as follows:
(Note, this is not a list of command you should run in order to get the system running.) 
@xref{Important} and @xref{Quick-start Checklist}.

@table @code
@item ./gradlew javadoc
JavaDocs end up in the @code{build/docs/javadoc} directory.
@item ./gradlew war
The WAR file ends up in the @code{build/libs} directory.  
This file can be just placed in the webapps directory of tomcat to deploy.
@item ./gradlew explodedWar
This builds a set of files that would have been placed in the deployment WAR file.  It is used, when needed,
to run the entire system from a local directory.  The files are placed in the @code{build/exploded/kiss}
directory.
@item ./gradlew clean
Removes all the generated files.
@item ./gradlew tomcatRun
Runs a sample server serving up the application.  The system will be
available from your local browser at
@code{http://localhost:8080}.  Use @code{^C} to exit.
@item ./gradlew tomcatRunDebug
Runs a sample server serving up the application.  The system will be
available from your local browser at
@code{http://localhost:8080}.  Use @code{^C} to exit. Once @emph{Kiss} is running in debug mode,
you may debug the application by attaching to the running @emph{tomcat} server at port 5005.
@end table

@subsection IntelliJ

IntelliJ is a good IDE.  It is the only IDE that supports debugging
@emph{Groovy} as microservices.  The free, community version is adequate for developing
and debugging with @emph{Kiss}.  IntelliJ should be configured with
support for @code{Gradle}, @code{tomcat}, and @code{Groovy}.  If using the Community Edition, all of the install defaults are correct. 

The following should be done for the Community or Ultimate Editions:

@enumerate
@item
Get a fresh clone of @emph{Kiss}
@item
When you first enter IntelliJ, select @code{Import Project} and select the root of your @emph{Kiss} clone.
@item
Select Import project from external model, then select @emph{Gradle}.
@item
Accept all defaults
@end enumerate

In order to do front-end and back-end development at the same time, a dual server setup where
the front-end and back-end are served by different servers is recommended.  @xref{Front-end Development}
This section documents the back-end portion.

@subsection Community Edition Too

The following is applicable to both the Community Edition and the Ultimate Editions of IntelliJ.

In the upper right hand corner of the IDE there is a tab labeled @emph{Gradle}.  When that tab is expanded you may execute @emph{Gradle} tasks that you select.
Expand @emph{Tasks} and then @emph{gretty}.  The main task you will be interested in is @emph{tomcatStartDebug}.  That task starts @emph{Kiss}
in debug mode suitable for debugging the back-end portion of @emph{Kiss}.  Double-clicking on that task builds the system and launched the integrated @emph{tomcat}
server.  

When the output windows says ``Listening for transport dt_socket at address: 5005'' then select @emph{Run} an then @emph{Attach to process...}.
A window asking you to select what to attach to will appear.  It will only contain a single item.  Select it. You can then go to the following URL
on your browser:  @uref{http://localhost:8080}  You can now debug the application.

Note two tabs on the bottom of the IDE labeled @emph{Run} and @emph{Debug}.  These two tabs are very useful during debugging.



@subsubsection Ultimate Edition

Additionally, the following steps may be used to simplify debugging when using the @emph{Ultimate} (paid) version of @emph{IntelliJ}:



@enumerate
@item
Go to Run / Edit Configurations
@item
Click the ``+'' in the upper left corner
@item
Select @emph{Tomcat Server} / @emph{Local}
@item
Change the Name to Kiss
@item
Click the Deployment tab
@item
Click the ``+'' in the upper right quadrant of the screen (not the upper left)
@item
Select @emph{Artifact} and then @emph{Kiss.war} (exploded)
@item
Set ``Application context:'' to ``/''
@item
Click "OK"
@end enumerate

To run the app in debug mode just click the green bug at the top of the screen.  This will run the front-end and back-end 
in a debug mode.  Although this is the recommended setup for the back-end portion during development, it is not recommended
for front-end development.  In order to do front-end and back-end development at the same time, a dual server setup where
the front-end and back-end are served by different servers is recommended.    @xref{Front-end Development}

@subsection NetBeans

The @emph{Kiss} system works with NetBeans.  The only shortcoming is
that @emph{Groovy} code can not be debugged.  These instructions are
for NetBeans 8.2.  These instructions need only be done once.

When installing NetBeans, be sure to select the following Runtimes:

@itemize @bullet
@item
Java EE
@item
HTML5 / JavaScript
@item
Groovy
@item
Apache Tomcat
@end itemize

Go into NetBeans and select @emph{File} and @emph{Open Project}.
Select the root of the @emph{Kiss} project.  From the @emph{Projects}
pane, you can right-click on the project root.  You will then see an
option labeled @emph{Tasks}.  That is where you can execute
@emph{Gradle} tasks.  To build and debug the application, select
@emph{tomcat} and then @emph{tomcatRunDebug}.  You will then see a
message ``Listening for transport dt_socket at address: 5005''.  From
the top menu, select @emph{Debug} and then @emph{Attach Debugger}.
The defaults should be correct.  Accept the popup.  You may then go to
the following URL on your browser: @uref{http://localhost:8080}
You can now debug the application.

The @emph{Output} tab on the bottom will display helpful information.


Although this is the recommended setup for the back-end portion during
development, it is not recommended for front-end development.  In
order to do front-end and back-end development at the same time, a
dual server setup where the front-end and back-end are served by
different servers is recommended.  @xref{Front-end Development}



@subsection eclipse, VSCode, and Others

Other IDE's that support @emph{Gradle} should work fine.  In
particular, @emph{VSCode} and @emph{eclipse} have been tested and are
known to work.

@section What Do I Do With It Now?

What you have at this point is the beginnings of your new application.
@emph{Kiss} is provided as a running and deployable system.  It is
expected that you would modify what's here to suit your application
needs.

Besides this narrative, you would need the JavaDocs located under the
@code{build/docs/javadoc} directory for API specific documentation.
(Which you must generate via @emph{Gradle} as described. @xref{javadoc,,Creating JavaDocs}.
