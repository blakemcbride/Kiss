@c -*-texinfo-*-

@c  Copyright (c) 2018 Blake McBride
@c  All rights reserved.
@c
@c  Redistribution and use in source and binary forms, with or without
@c  modification, are permitted provided that the following conditions are
@c  met:
@c
@c  1. Redistributions of source code must retain the above copyright
@c  notice, this list of conditions and the following disclaimer.
@c
@c  2. Redistributions in binary form must reproduce the above copyright
@c  notice, this list of conditions and the following disclaimer in the
@c  documentation and/or other materials provided with the distribution.
@c
@c  THIS SOFTWARE IS PROVIDED BY THE COPYRIGHT HOLDERS AND CONTRIBUTORS
@c  "AS IS" AND ANY EXPRESS OR IMPLIED WARRANTIES, INCLUDING, BUT NOT
@c  LIMITED TO, THE IMPLIED WARRANTIES OF MERCHANTABILITY AND FITNESS FOR
@c  A PARTICULAR PURPOSE ARE DISCLAIMED. IN NO EVENT SHALL THE COPYRIGHT
@c  HOLDER OR CONTRIBUTORS BE LIABLE FOR ANY DIRECT, INDIRECT, INCIDENTAL,
@c  SPECIAL, EXEMPLARY, OR CONSEQUENTIAL DAMAGES (INCLUDING, BUT NOT
@c  LIMITED TO, PROCUREMENT OF SUBSTITUTE GOODS OR SERVICES; LOSS OF USE,
@c  DATA, OR PROFITS; OR BUSINESS INTERRUPTION) HOWEVER CAUSED AND ON ANY
@c  THEORY OF LIABILITY, WHETHER IN CONTRACT, STRICT LIABILITY, OR TORT
@c  (INCLUDING NEGLIGENCE OR OTHERWISE) ARISING IN ANY WAY OUT OF THE USE
@c  OF THIS SOFTWARE, EVEN IF ADVISED OF THE POSSIBILITY OF SUCH DAMAGE.

@node Back-end API

@html
@include style.css
@end html

@chapter Back-end API

In addition to the full API provided by the Java system and any additional
JAR files you add, @emph{Kiss} comes with an API that assists with the
development of business application with @emph{Kiss}.  These 
API's may be broadly grouped as follows:

@enumerate
@item
Database API
@item
JSON API
@item
Utilities
@end enumerate

An overview of these APIs is contained in this chapter.  Detailed
documentation is contained in the JavaDocs.  @xref{javadoc,,Creating JavaDocs}.

@section Database API

@emph{Kiss} comes with a powerful library for accessing SQL databases.
Code for this is located under @code{org.kissweb.database} It is
currently being used in production environments.  This API provides
the following benefits:

@itemize @bullet
@item
Automatic connection and statement pooling
@item
Vastly simpler API than bare JDBC
@item
Handling of parameterized arguments
@item 
Auto generation of SQL for single record adds, edits, and deletions
@item
Auto handling for cases of cursor interference on nested queries
@item
Supports transactions out-of-the-box
@end itemize

As shipped, this library supports PostgreSQL, Microsoft Server,
Oracle, MySQL, and SQLite.

The detailed documentation for the database utilities is in the
JavaDocs which you must generate. (@xref{javadoc,,Creating JavaDocs}.)
This section provides an overview.

The @emph{Kiss} database routines revolve around four main classes as follows:

@table @code
@item Connection
This represents a connection to an SQL database.
@item Command
This represents a single action or command against the database.
@item Cursor
If the action is a @code{select}, the @code{Cursor} represents a
pointer into the result set.
@item Record
This class represents a single row within a table or result set.
@end table

The @code{Connection} class contains several convenience methods that
are used in simple cases where only a single action is being
performed.  These methods should not be used when multiple
simultaneous actions are taking place at once (by that single thread).
This issue is not a problem in multi-user or multi-threaded
situations.  It is only a problem when a single thread is doing one
action while another action is still open.

You will notice that your REST services are passed a @emph{Connection}
argument.  @emph{Kiss} automatically forms a unique connection for
each REST service call and closes it when the call is done.
Therefore, you will not normally need to create your own connection.
Additionally, @emph{Kiss} automatically starts a new transaction with
each REST service and commits it when the service is done.  However,
if the service throws any exception, the transaction is rolled back
instead.

You would not normally write SQL for single record adds, updates, and
deletes.  Using the @emph{Record} API, @code{Kiss} automatically
generates these statements for you.

In addition to the above, these utilities provide full support for transactions and parameters.

@section Microservices

Microservices are classes that may be added, changed, or deleted while
the system is running.  In spite of this, however, all microservices
are fully compiled and run at full speed.  This has two major advantages.

First, in a development environment, all development may be done without 
the need to bring the development server down, rebuild, re-deploy, and reboot
the development server.  This means development time is significantly reduced.
Additionally, IDE debuggers function in an environment such as this, so
the debugging process may proceed normally.

Second, in a production environment, the system may be upgraded
without disrupting existing users on the system.  Of course, users
actively using the exact features you have just changed could be
affected (if their front-end and back-end do not agree).
A simple re-try would put everything back in sync.

Kiss microservices are on a class basis.  What that means is that a
microservice is always a single and whole class.  You cannot have more
than one class in a microservice.  Microservices can call
core components of the system just as regular methods can.  However,
if one wishes to have one microservice call another microservice,
the calling mechanism is a little more clunky.  However, using this
clunkier mechanism retains all of the dynamic features of the system.

Defining microservices in Kiss involves defining a normal class just
as you would write a class in any circumstance.  No special
configuration, wiring, declarations, or additional steps are required.
Microservice additions, changes, and deletions take effect as soon as
you save the source file.  All microservices are compiled at runtime
by the system automatically, so there is no compilation step that
you need to perform.

The only caveat to the above is that remote microservices (described
below) expect a certain method signature (standard arguments).  This
is only to assure that the front-end and back-end can communicate as
expected.  Local microservices do not have this requirement.

@subsection Microservice Language

In Kiss, microservices can be written in Java, Groovy, or Common Lisp.
However, Groovy microservices have been used exclusively in all
current environments that we are aware of.  Therefore, Groovy
microservices are best tested.

The reasons Groovy was used are as follows:

@enumerate
@item
Groovy was the easiest and most natural to implement.
@item
Groovy runs as fast as Java and has full and natural access to all 
Java facilities.
@item
The actual loading of Groovy services is the fastest of the three languages.
@item
Groovy is largely a super-set of Java so if you know Java, you basically
know Groovy.  Learning of additional Groovy facilities can occur over
time.
@item
Groovy offers a small number of conveniences over Java.
@end enumerate

Having said all this, however, all three languages are fully supported
and there are no known bugs in any of the supported languages.

@subsection Types of Microservices

There are two types of microservices as follows:

@enumerate
@item
Remote (REST) Services
@item
Local Services
@end enumerate

Remote services are generally called by a (likely JavaScript)
front-end or a remote client typically over @emph{HTTP} or
@emph{HTTPS}.  Local services are services that are called locally,
from within the system as, for example, one microservice calling a
method in a different microservice.

@subsection Remote Microservices

In @emph{Kiss}, remote microservices appear to the outside world as
typical @emph{REST} services.  They may be called by a
@emph{JavaScript} front-end, web service client, or any other facility
that can talk to @emph{REST} services.

@emph{Kiss} @emph{REST} services are asynchronous @emph{HTTP} or
@emph{HTTPS} services.  Internally they are processed utilizing a
thread pool to assure optimal CPU utilization.  The number of threads
in the thread pool is controlled by a configuration parameter given in
the @code{KissInit.groovy} file.

@emph{Kiss} comes with @emph{JavaScript} code so that @code{Kiss}
@code{REST} services can be easily accessed from a typical front-end.
This code resides in a single file and can and has been used by
alternative front-ends such as @emph{Angular} and @code{React}.

There is no need to handle authentication.  @emph{Kiss} handles that
automatically.  So, when your web method is called, you know who
they are and that they have been authenticated.

@subsection Defining Remote Services

Remote REST microservices typically reside under the
@code{backend/services} directory.  You can organize them any way you
like.  A microservice is equivalent to a class.  The class name is the
microservice name.  The methods in that class that have a certain
signature (a particular set of arguments) are the web methods.

A REST web microservice has the following signature:

@example
void myMethod(JSONObject injson, 
              JSONObject outjson, 
              Connection db, 
              ProcessServlet servlet) @{
...
@}
@end example

@table @samp
@item injson
This is a @code{JSON} object that contains all of the arguments passed
in from the front-end.
@item outjson
This is a @code{JSON} object that will contain all of the results
sent back to the front-end.  Whatever is put in this object gets sent back to the front-end.
@item db
This is a database connection that can be used to access the SQL database.  The
connection is unique and independent of all other services.
@item servlet
This object provides access to various system facilities uniquely related
to this call.
@end table

@xref{json,,JSON} and the @code{JavaDoc} for additional information.


@subsection Local Microservices

Local microservices are simply regular classes.  They typically reside
in the @code{backend/services} directory organized any way you like.
Although methods within a particular microservice/class can call each
other in the normal way, there is an extra step required for one
microservice to call a method in a different microservice.  One of the
reasons for this is so the system can be certain the latest version of
the service is loaded and that it is fully compiled before you attempt
to use it.

In Groovy, a method in one class/microservice can call a method
in a different class/microservice via the following methods.

@itemize @bullet
@item
GroovyService.run
@item
GroovyService.getMethod
@end itemize

See the @emph{Javadoc}.

@anchor{json}@section JSON

The first two arguments to all REST methods are @code{injson} and
@code{outjson}.  @code{injson} is a @code{JSONObject} that contains
the data passed in @emph{from} the front-end.  @code{outjson} is a
pre-initialized, empty @code{JSONObject} that will be @emph{returned
to} the front-end.  The REST service should read the data passed in
from @code{injson}, perform any needed processes, and put the result
into @code{outjson} to be returned to the front-end.

A modified version of a publicly available JSON Java package is
included and used to access JSON from the front-end and create JSON to
return to the front-end.  This package has many methods but only a few are
commonly used.

There are two main data types of interest.  They are @code{JSONObject} and
@code{JSONArray}.  They hold the JSON types indicated by their names.

@noindent
Command useful for getting data out of @code{injson}:

@itemize @bullet
@item
JSONObject.has(String key)
@item
JSONObject.getString(String key)
@item
JSONObject.getBoolean(String key)
@item
JSONObject.getInt(String key)
@item
JSONObject.getLong(String key)
@item
JSONObject.getDouble(String key)
@item
JSONObject.getFloat(String key)
@item
JSONObject.getJSONArray(String key)
@item
JSONArray.length()
@item
JSONArray.getString(int index)
@item
JSONArray.getBoolean(int index)
@item
JSONArray.getInt(int index)
@item
JSONArray.getLong(int index)
@item
JSONArray.getFloat(int index)
@item
JSONArray.getDouble(int index)
@item
JSONArray.getString(int index)
@item
JSONArray.getJSONArray(int index)
@end itemize

@noindent
Again, @code{outjson} is supplied as a pre-initialized, but empty,
@code{JSONObject}.  It is up to the REST service code to populate it
with the return data.  Useful JSON utilities include the following:

@itemize @bullet
@item
JSONObject.put(String label, Object data)
@item
new JSONObject()
@item
new JSONArray()
@item
JSONArray.put(Object obj)
@end itemize

@section Large Language Models

@emph{Kiss} provides a high-level interface to both the @emph{Ollama} large language model (LLM) and the OpenAI (chatGPT) server. This makes it easy for you to add LLM features to your application.

@subsection Ollama

Before this can function, however, you must have access to an @emph{Ollama} server, either running on your local machine or a remote machine.

@itemize @bullet
@item
See @uref{https://ollama.com} for information about installing the @emph{Ollama} server.
@item
The @emph{Kiss} class @code{org.kissweb.llm.Ollama} implements the interface.
@item
@emph{Kiss} comes with a complete demo of this functionality.
@end itemize

@subsection OpenAI (chatGPT)

Unlike the @emph{Ollama} server, which runs locally, the @emph{chatGPT}
server runs on the cloud-based @emph{OpenAI} server.  In order to use
this server, you must have an account with them.  Once you get an
account, you will need an API Key to interface with their server.
Once you have this API Key, you may query @emph{chatGPT}.

The single @emph{KISS} class that interfaces with @emph{OpenAI} is
named @emph{OpenAI}.  

Typical usage would look as follows:

@example
    String apiKey = "xxxxxxxxxxxxxxx";
    OpenAI chatGPT = new OpenAI(apiKey, "gpt-3.5-turbo")
    String result = chatGPT.send("Who is president Bush?")
@end example

See that class for documentation.

@section Utilities

@emph{Kiss} includes an ever-growing set of utilities to help deal with common tasks.
These utilities are located under the @code{src/main/core/org/kissweb} directory and have names such as
@code{DateTime.java}, @code{NumberFormat}, etc.  These utilities
are documented in the JavaDocs.
