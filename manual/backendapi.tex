@c -*-texinfo-*-

@c  Copyright (c) 2018 Blake McBride
@c  All rights reserved.
@c
@c  Redistribution and use in source and binary forms, with or without
@c  modification, are permitted provided that the following conditions are
@c  met:
@c
@c  1. Redistributions of source code must retain the above copyright
@c  notice, this list of conditions and the following disclaimer.
@c
@c  2. Redistributions in binary form must reproduce the above copyright
@c  notice, this list of conditions and the following disclaimer in the
@c  documentation and/or other materials provided with the distribution.
@c
@c  THIS SOFTWARE IS PROVIDED BY THE COPYRIGHT HOLDERS AND CONTRIBUTORS
@c  "AS IS" AND ANY EXPRESS OR IMPLIED WARRANTIES, INCLUDING, BUT NOT
@c  LIMITED TO, THE IMPLIED WARRANTIES OF MERCHANTABILITY AND FITNESS FOR
@c  A PARTICULAR PURPOSE ARE DISCLAIMED. IN NO EVENT SHALL THE COPYRIGHT
@c  HOLDER OR CONTRIBUTORS BE LIABLE FOR ANY DIRECT, INDIRECT, INCIDENTAL,
@c  SPECIAL, EXEMPLARY, OR CONSEQUENTIAL DAMAGES (INCLUDING, BUT NOT
@c  LIMITED TO, PROCUREMENT OF SUBSTITUTE GOODS OR SERVICES; LOSS OF USE,
@c  DATA, OR PROFITS; OR BUSINESS INTERRUPTION) HOWEVER CAUSED AND ON ANY
@c  THEORY OF LIABILITY, WHETHER IN CONTRACT, STRICT LIABILITY, OR TORT
@c  (INCLUDING NEGLIGENCE OR OTHERWISE) ARISING IN ANY WAY OUT OF THE USE
@c  OF THIS SOFTWARE, EVEN IF ADVISED OF THE POSSIBILITY OF SUCH DAMAGE.


@chapter Back-end API

@section REST Services

All communications between the front-end and back-end occur over REST services you define.  Each REST service exists in its own, single file.

As architected, each directory under @code{src/main/application} represents a single REST service.  Each class / file under that directory represents
a web method.  The name of the class is the name of the web method.

Each class must have a static main method.  This is the method that the @emph{Kiss} core calls.  This method is passed four argument as follows:

@table @code
@item JSONObject injson
This represents the data that came from the front-end.
@item JSONObject outjson
This represents the data being returned to the front-end.
@item Connection db
This is a pre-opened connection to the defined database (defined in @code{application/KissInit.groovy})
@item MainServlet servlet
This is a rarely used servlet context argument.
@end table

Basically what happens is:

@enumerate
@item
The front-end make a REST service call.
@item
The @code{Kiss} back-end receives the request.
@item
The user gets authenticated.
@item
A new database connection is formed.
@item
The requested web services is loaded (and compiled if needed).
@item
The @code{outjson} object that is filled in by the web service is returned to the front-end.
@end enumerate

Of course during this process, @code{Kiss} handles many possible error conditions.

@section Database API


@emph{Kiss} comes with a powerful library for accessing SQL databases.
Code for this is located under @code{org.kissweb.database} It is
currently being used in production environments.  This API provides
the following benefits:

@itemize @bullet
@item
Automatic connection and statement pooling
@item
Vastly simpler API than bare JDBC
@item
Handling of parameterized arguments
@item 
Auto generation of SQL for single record adds, edits, and deletions
@item
Auto handling for cases of cursor interference on nested queries
@item
Supports transactions out-of-the-box
@end itemize

As shipped, this library supports PostgreSQL, Microsoft Server, Oracle, MySQL, and SQLite.

The detailed documentation for the database utilities are the
accompanying JavaDocs.  This section provides an overview.

The @emph{Kiss} database routines revolve around four main classes as follows:

@table @code
@item Connection
This represents a connection to an SQL database.
@item Command
This represents a single action or command against the database.
@item Cursor
If the action is a @code{select}, the @code{Cursor} represents a pointer into the result set.
@item Record
This class represents a single row within a table or result set.
@end table

The @code{Connection} class contains several convenience methods that
are used in simple cases where only a single action is being
performed.  These methods should not be used when multiple
simultaneous actions are taking place at once (by that single thread).
This issue is not a problem in multi-user or multi-threaded
situations.  It is only a problem when a single thread is doing one
action while another action is still open.

You will notice that your REST services are passed a @emph{Connection}
argument.  @emph{Kiss} automatically forms a unique connection for
each REST service call and closes it when the call is done.
Therefore, you will not normally need to create your own connection.

You would not normally write SQL for single record adds, updates, and
deletes.  Using the @emph{Record} API, @code{Kiss} automatically
generates these statements for you.

In addition to the above, these utilities provide full support for transactions and parameters.

@section Utilities
