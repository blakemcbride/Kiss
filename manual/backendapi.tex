@c -*-texinfo-*-

@c  Copyright (c) 2018 Blake McBride
@c  All rights reserved.
@c
@c  Redistribution and use in source and binary forms, with or without
@c  modification, are permitted provided that the following conditions are
@c  met:
@c
@c  1. Redistributions of source code must retain the above copyright
@c  notice, this list of conditions and the following disclaimer.
@c
@c  2. Redistributions in binary form must reproduce the above copyright
@c  notice, this list of conditions and the following disclaimer in the
@c  documentation and/or other materials provided with the distribution.
@c
@c  THIS SOFTWARE IS PROVIDED BY THE COPYRIGHT HOLDERS AND CONTRIBUTORS
@c  "AS IS" AND ANY EXPRESS OR IMPLIED WARRANTIES, INCLUDING, BUT NOT
@c  LIMITED TO, THE IMPLIED WARRANTIES OF MERCHANTABILITY AND FITNESS FOR
@c  A PARTICULAR PURPOSE ARE DISCLAIMED. IN NO EVENT SHALL THE COPYRIGHT
@c  HOLDER OR CONTRIBUTORS BE LIABLE FOR ANY DIRECT, INDIRECT, INCIDENTAL,
@c  SPECIAL, EXEMPLARY, OR CONSEQUENTIAL DAMAGES (INCLUDING, BUT NOT
@c  LIMITED TO, PROCUREMENT OF SUBSTITUTE GOODS OR SERVICES; LOSS OF USE,
@c  DATA, OR PROFITS; OR BUSINESS INTERRUPTION) HOWEVER CAUSED AND ON ANY
@c  THEORY OF LIABILITY, WHETHER IN CONTRACT, STRICT LIABILITY, OR TORT
@c  (INCLUDING NEGLIGENCE OR OTHERWISE) ARISING IN ANY WAY OUT OF THE USE
@c  OF THIS SOFTWARE, EVEN IF ADVISED OF THE POSSIBILITY OF SUCH DAMAGE.


@chapter Back-end API

In addition to the full API provided by the Java system, and any additional
JAR files you add, @emph{Kiss} comes with an API that assists with the
development of business application with @emph{Kiss}.  These 
API's may be broadly grouped as follows:

@enumerate
@item
Database API
@item
JSON API
@item
Utilities
@end enumerate

An overview of these API's is contained in this chapter.  Detailed
documentation is contained in the JavaDocs.  @xref{javadoc,,Creating JavaDocs}.

@section Database API

@emph{Kiss} comes with a powerful library for accessing SQL databases.
Code for this is located under @code{org.kissweb.database} It is
currently being used in production environments.  This API provides
the following benefits:

@itemize @bullet
@item
Automatic connection and statement pooling
@item
Vastly simpler API than bare JDBC
@item
Handling of parameterized arguments
@item 
Auto generation of SQL for single record adds, edits, and deletions
@item
Auto handling for cases of cursor interference on nested queries
@item
Supports transactions out-of-the-box
@end itemize

As shipped, this library supports PostgreSQL, Microsoft Server, Oracle, MySQL, and SQLite.

The detailed documentation for the database utilities are in the
JavaDocs which you must generate. (@xref{javadoc,,Creating JavaDocs}.) This section provides an overview.

The @emph{Kiss} database routines revolve around four main classes as follows:

@table @code
@item Connection
This represents a connection to an SQL database.
@item Command
This represents a single action or command against the database.
@item Cursor
If the action is a @code{select}, the @code{Cursor} represents a pointer into the result set.
@item Record
This class represents a single row within a table or result set.
@end table

The @code{Connection} class contains several convenience methods that
are used in simple cases where only a single action is being
performed.  These methods should not be used when multiple
simultaneous actions are taking place at once (by that single thread).
This issue is not a problem in multi-user or multi-threaded
situations.  It is only a problem when a single thread is doing one
action while another action is still open.

You will notice that your REST services are passed a @emph{Connection}
argument.  @emph{Kiss} automatically forms a unique connection for
each REST service call and closes it when the call is done.
Therefore, you will not normally need to create your own connection.
Additionally, @emph{Kiss} automatically starts a new transaction with
each REST service and commits it when the service is done.  However,
if the service throws any exception, the transaction is rolled back
instead.

You would not normally write SQL for single record adds, updates, and
deletes.  Using the @emph{Record} API, @code{Kiss} automatically
generates these statements for you.

In addition to the above, these utilities provide full support for transactions and parameters.

@section JSON

The first two arguments to all REST methods is @code{injson} and
@code{outjson}.  @code{injson} is a @code{JSONObject} that contains
the data passed in @emph{from} the front-end.  @code{outjson} is a
pre-initialized, empty @code{JSONObject} that will be @emph{returned
to} the front-end.  The rest service should read the data passed in
from @code{injson}, perform any needed processes, and put the result
into @code{outjson} to be returned to the front-end.

A modified version of a publicly available JSON Java package is
included and used to access JSON from the front-end and create JSON to
return to the front-end.  This package has many methods but only a few are
commonly used.

There are two main data type of interest.  They are @code{JSONObject} and
@code{JSONArray}.  They hold the JSON types indicated by their names.

@noindent
Command useful for getting data out of @code{injson}:

@itemize @bullet
@item
JSONObject.has(String key)
@item
JSONObject.getString(String key)
@item
JSONObject.getBoolean(String key)
@item
JSONObject.getInt(String key)
@item
JSONObject.getLong(String key)
@item
JSONObject.getDouble(String key)
@item
JSONObject.getFloat(String key)
@item
JSONObject.getJSONArray(String key)
@item
JSONArray.length()
@item
JSONArray.getString(int index)
@item
JSONArray.getBoolean(int index)
@item
JSONArray.getInt(int index)
@item
JSONArray.getLong(int index)
@item
JSONArray.getFloat(int index)
@item
JSONArray.getDouble(int index)
@item
JSONArray.getString(int index)
@item
JSONArray.getJSONArray(int index)
@end itemize

@noindent
Again, @code{outjson} is supplied as a pre-initialized, but empty,
@code{JSONObject}.  It is up to the REST service code to populate it
with the return data.  Useful JSON utilities include the following:

@itemize @bullet
@item
JSONObject.put(String label, Object data)
@item
new JSONObject()
@item
new JSONArray()
@item
JSONArray.put(Object obj)
@end itemize


@section Utilities
