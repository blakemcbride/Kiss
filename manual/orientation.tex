@c -*-texinfo-*-

@c  Copyright (c) 2018 Blake McBride
@c  All rights reserved.
@c
@c  Redistribution and use in source and binary forms, with or without
@c  modification, are permitted provided that the following conditions are
@c  met:
@c
@c  1. Redistributions of source code must retain the above copyright
@c  notice, this list of conditions and the following disclaimer.
@c
@c  2. Redistributions in binary form must reproduce the above copyright
@c  notice, this list of conditions and the following disclaimer in the
@c  documentation and/or other materials provided with the distribution.
@c
@c  THIS SOFTWARE IS PROVIDED BY THE COPYRIGHT HOLDERS AND CONTRIBUTORS
@c  "AS IS" AND ANY EXPRESS OR IMPLIED WARRANTIES, INCLUDING, BUT NOT
@c  LIMITED TO, THE IMPLIED WARRANTIES OF MERCHANTABILITY AND FITNESS FOR
@c  A PARTICULAR PURPOSE ARE DISCLAIMED. IN NO EVENT SHALL THE COPYRIGHT
@c  HOLDER OR CONTRIBUTORS BE LIABLE FOR ANY DIRECT, INDIRECT, INCIDENTAL,
@c  SPECIAL, EXEMPLARY, OR CONSEQUENTIAL DAMAGES (INCLUDING, BUT NOT
@c  LIMITED TO, PROCUREMENT OF SUBSTITUTE GOODS OR SERVICES; LOSS OF USE,
@c  DATA, OR PROFITS; OR BUSINESS INTERRUPTION) HOWEVER CAUSED AND ON ANY
@c  THEORY OF LIABILITY, WHETHER IN CONTRACT, STRICT LIABILITY, OR TORT
@c  (INCLUDING NEGLIGENCE OR OTHERWISE) ARISING IN ANY WAY OUT OF THE USE
@c  OF THIS SOFTWARE, EVEN IF ADVISED OF THE POSSIBILITY OF SUCH DAMAGE.


@node Orientation

@html
@include style.css
@end html

@chapter Orientation


The entire source code comes with the system and is convenient when
debugging, however, only a few areas in the system would normally be
of concern when building an application as follows.


@section Back-end Application Files

@table @code
@item src/main/backend/KissInit.groovy
This file is used to configure the system. @xref{Setup and Configuration}
@item src/main/backend
All other files under this directory represent the application
back-end.  All the files are used and distributed in source form.  The
Kiss system compiles them at runtime but does not save the compiled form.
Updates to files under this directory take affect
immediately on a running system.
@end table


@section Front-end Application Files

Files under the @code{src/main/frontend} directory represent the
front-end of the application.

All files under the @code{src/main/frontend/kiss} directory are part of
the @emph{Kiss} system and would normally not need to be touched.

@code{index.html} and @code{index.js} are also part of the @emph{Kiss}
system and aren't normally modified.

@code{login.html} and @code{login.js} represent the user login page
and would be modified to suit your needs.

Other directories such as @code{page1} represent other user pages and
would be the application specific screens you create.  The included
@code{page1} directory is only an example page.

@section Database

@emph{Kiss} supports Microsoft SQL Server, Oracle, PostgreSQL, MySQL, and
SQLite.

As shipped, @emph{Kiss} comes configured with an embedded SQLite
server and database.  While this is fine for a demo or small
application, a real database should be configured for real use.
The included database is located in the @code{backend} directory
and is named @code{DB.sqlite}

Although @emph{Kiss} has no preferred database, PostgreSQL is strongly
recommended because it is free, full featured, fast, rock solid, and
portable on all major platforms.

@xref{Setup and Configuration}


@section Single Page Application

@emph{Kiss} applications are single page applications in the sense
that there is a single @code{<body>} tag and all other pages
essentially get placed into that tag on a single page.  However,
@emph{Kiss} is not a single page application in the sense that the
entire application gets loaded with a single @code{GET} request.  This
doesn't make sense for a large business application in which many
hundreds of pages may exist.  @emph{Kiss} lazy-loads pages as they are
used, and except for browser cache, eliminates them once another page
is loaded.

@section Controlling Browser Cache

The user's browser cache can be controlled from the file
@code{src/main/frontend/index.html} In that file you will see two lines
that look as follows:

@example
var softwareVersion = "1";  // version of the entire system
var controlCache = false;   // normally true but use false during 
                            // debugging
@end example

If @code{controlCache} is set to @code{true}, each time @code{softwareVersion}
is incremented all users starting the application will be forced to
load new code from the server and not use their browser's cache.  Once
they download the new version, normal browser cache activity will
occur.


@anchor{javadoc} @section Creating JavaDocs

JavaDocs for the @emph{Kiss} system will need to be created.  It is
created from the command line by issuing the following command:

@example
./bld javadoc              [Linux, Mac, BSD, etc.]
    -or-
bld javadoc                [Windows]
@end example

@noindent
The JavaDoc files end up in the @code{work/javadoc} directory.


@section Deploying A Kiss Application

The only file needed to deploy the application is @code{Kiss.war} It
can be built by typing @code{./bld war} at a command prompt.
@code{Kiss.war} ends up in the @code{work} directory.  If you
have your IDE create the @code{Kiss.war} file, it will likely not
work.  The @emph{Kiss} system requires a special build process because
application files are distributed in source form.  Therefore, @code{bld}
should be used to create the production WAR file.

If using @emph{tomcat}, @code{Kiss.war} should be placed in the
@code{webapps} directory.  When @emph{tomcat} starts, it will see the
file, unpack it, and run it.  The application will be available at
@code{[HOST]/Kiss}

Renaming @code{Kiss.war} to @code{ABC.war}, for example, will cause
the application path to change to @code{[HOST]/ABC}

@section Learning The System

In order to start getting a feel for how @code{Kiss} applications
function, in terms of the back-end, look at files in the
@code{src/main/backend/services} directory.  With @code{Kiss} you can
develop applications in several different languages.  The @code{services}
example shows the same code in all of the supported languages.

In terms of the front-end, see the example files under @code{src/main/frontend/page1}
