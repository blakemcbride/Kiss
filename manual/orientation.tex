@c -*-texinfo-*-

@c  Copyright (c) 2018 Blake McBride
@c  All rights reserved.
@c
@c  Redistribution and use in source and binary forms, with or without
@c  modification, are permitted provided that the following conditions are
@c  met:
@c
@c  1. Redistributions of source code must retain the above copyright
@c  notice, this list of conditions and the following disclaimer.
@c
@c  2. Redistributions in binary form must reproduce the above copyright
@c  notice, this list of conditions and the following disclaimer in the
@c  documentation and/or other materials provided with the distribution.
@c
@c  THIS SOFTWARE IS PROVIDED BY THE COPYRIGHT HOLDERS AND CONTRIBUTORS
@c  "AS IS" AND ANY EXPRESS OR IMPLIED WARRANTIES, INCLUDING, BUT NOT
@c  LIMITED TO, THE IMPLIED WARRANTIES OF MERCHANTABILITY AND FITNESS FOR
@c  A PARTICULAR PURPOSE ARE DISCLAIMED. IN NO EVENT SHALL THE COPYRIGHT
@c  HOLDER OR CONTRIBUTORS BE LIABLE FOR ANY DIRECT, INDIRECT, INCIDENTAL,
@c  SPECIAL, EXEMPLARY, OR CONSEQUENTIAL DAMAGES (INCLUDING, BUT NOT
@c  LIMITED TO, PROCUREMENT OF SUBSTITUTE GOODS OR SERVICES; LOSS OF USE,
@c  DATA, OR PROFITS; OR BUSINESS INTERRUPTION) HOWEVER CAUSED AND ON ANY
@c  THEORY OF LIABILITY, WHETHER IN CONTRACT, STRICT LIABILITY, OR TORT
@c  (INCLUDING NEGLIGENCE OR OTHERWISE) ARISING IN ANY WAY OUT OF THE USE
@c  OF THIS SOFTWARE, EVEN IF ADVISED OF THE POSSIBILITY OF SUCH DAMAGE.


@node Orientation

@html
@include style.css
@end html

@chapter Orientation


The entire source code comes with the system and is convenient when
debugging.  @emph{Kiss} has been designed to segregate @emph{Kiss}
framework code from your application code for two reasons:

@enumerate
@item
It makes it easy to differentiate framework code from your application code minimizing
the code you have to look at when developing your application.
@item
It makes upgrading the @emph{Kiss} framework simple and accurate.
@end enumerate

Naturally, in order for this to work, it is important that you not change framework code
and put all of your application code in the correct areas.

@section Back-end Application Files

Some back-end code is normal code that gets pre-compiled at build time,
as most systems do --- as the @emph{Kiss} core framework code is.
However, some @emph{Kiss} code (the majority of the application code)
is @emph{hot-loaded} at runtime.  What this means is that code that is
hot-loaded can be added, changed, or deleted on a running system.  
There is no need to bring the system down or rebuild anything. In a development
environment, development and debugging occur while the system is
running continuously.  There is no need to bring the system down or rebuild
anything.  A production system can be updated while the system is
running without the need to reboot the server or upset normal user use
of the system.

All back-end code resides in the following directories:


@table @code
@item src/main/backend
The vast majority of your application-specific code goes in this directory tree. 
All files in this directory tree (except @code{KissInit.groovy}) are hot-loaded. 

This directory contains the following:
@table @code
@item services
This is where all of your web services will be located.  You can structure sub-directories to it as needed.
Web services can call any core @emph{Kiss} code, @emph{jar} files, or @code{precompiled} code directly.
However, although web services can call other hot-loaded files located in the @code{backend} tree,
a special syntax is required.  There is an example of this in @code{services/MyGroovyService.groovy}.

Unlike other coding philosophies that prefer thin web services that
call core code, @emph{Kiss} strongly prefers thick web services that
contain the bulk of your application code.  This maximizes the use and
value of the hot-load facility.
@item scripts
This directory contains code that is not web services but common
application code that is to be hot-loaded.  However, there is a big
negative associated with this code.  While hot-loaded web services can
directly call @emph{Kiss} core code, @emph{jar} files, and
@code{precompiled} code, they can't call methods in the scripts
directory directly.  They must use a convoluted syntax enabling the
hot-loading of those methods. An example of this exists in the
@code{services/MyGroovyService.groovy}.


@item KissInit.groovy
This file is used to configure the system. It is read once upon system
startup and is not hot-loaded.  @xref{Setup and Configuration}

@item Login.groovy
This is the code that validates a user's login.  It is kept here because it is common for application 
customization of the login process.

@item CronTasks
Files in this directory support the ability to auto-start processes at
scheduled and recurring dates and times ala Unix @emph{cron}.  See the
files in that directory for documentation.


@end table

@item src/main/precompiled

Code in this directory tree is compiled at build time and not hot-loaded.  This directory tree is used to
store common application-specific code that doesn't change particularly often.

@item src/main/core
@emph{Kiss} core back-end framework code resides in this directory tree. Nothing in this directory tree should be 
modified by you because it will be overwritten when @emph{Kiss} is upgraded.  (@xref{updates})

@end table


@section Front-end Application Files

Files under the @code{src/main/frontend} directory represent the
front-end of the application.

All files under the @code{src/main/frontend/kiss} directory are part of
the @emph{Kiss} system and should not be modified.

@code{index.html} and @code{index.js} are also part of the @emph{Kiss}
system and aren't normally modified.  They contain code that ensures
that browsers load updated code.

@code{index.html} contains three important variables that assure that users correctly load 
front-end code that has been changed rather than using their cached version.  These variables are as follows:

@table @code
@item SystemInfo.softwareVersion
This can be set to any unique string.  When the system is in production use, if this string changes,
the browser will load new copies of all front-end files (which it will cache for future use).
If any front-end files are changed, this variable should be changed.  The users will get the new screens 
after they log out and back into the system.

@item SystemInfo.controlCache

Setting this variable to @code{true} tells the system to observe the value in @code{SystemInfo.softwareVersion}
and re-load the front-end screens if its value changes.

The value of this variable is often set to @code{false} during the development process to avoid double-loading
of certain files.   When debugging the front-end, be sure to disable the browser cache.

@item SystemInfo.releaseDate 
This variable is used to track the release date of a given front-end.  It is used for display purposes only.

@end table

In addition to differentiating between development and production environments, @code{index.js}
is also used to detect the user's device type (e.g., desktop, tablet, mobile) and, if desired, 
load different application screens.


@code{login.html} and @code{login.js} represent the user login page
and should be modified to suit your needs.

Other directories such as @code{page1} represent other user pages and
would be the application-specific screens you create.  The included
@code{page1} directory is only an example page.

@section Database

@emph{Kiss} supports Microsoft SQL Server, Oracle, PostgreSQL, MySQL, and
SQLite.

As shipped, @emph{Kiss} comes configured with an embedded SQLite
server and database.  While this is fine for a demo or small
application, a real database should be configured for real use.
The included database is located in the @code{backend} directory
and is named @code{DB.sqlite}

Although @emph{Kiss} has no preferred database, PostgreSQL is strongly
recommended because it is free, full-featured, fast, rock solid, and
portable on all major platforms.

@xref{Setup and Configuration}


@section Single Page Application

@emph{Kiss} applications are single-page applications in the sense
that there is a single @code{<body>} tag and all other pages
essentially get placed into that tag on a single page.  However,
@emph{Kiss} is not a single-page application in the sense that the
entire application gets loaded with a single @code{GET} request.  This
doesn't make sense for a large business application in which many
hundreds of pages may exist.  @emph{Kiss} lazy-loads pages as they are
used, and except for browser cache, eliminates them once another page
is loaded.

@section Controlling Browser Cache

As discussed previously, the user's browser cache can be controlled from the file
@code{src/main/frontend/index.html}. In that file, you will see two lines
that look as follows:

@example
SystemInfo.softwareVersion = "1";  // version of the entire system
SystemInfo.controlCache = false;   // normally true but use false during
                            // debugging
@end example

If @code{SystemInfo.controlCache} is set to @code{true}, each time @code{SystemInfo.softwareVersion}
is incremented, all users starting the application will be forced to
load new code from the server and not use their browser's cache.  Once
they download the new version, normal browser cache activity will
occur.

Users using the system when front-end files are changed will not get
the updated files until they log out and back into the system.  In a
development environment with browser cache disabled, the developer
will see all screen changes upon reload, regardless of the variable settings.
In fact, it is better to set @code{SystemInfo.controlCache} to @code{false}
in a development environment to avoid a needless duplicate load of @code{index.html}.


@anchor{javadoc} @section Creating JavaDocs

JavaDocs for the @emph{Kiss} system will need to be created.  They are
created from the command line by issuing the following command:

@example
./bld javadoc              [Linux, macOS, BSD, etc.]
    -or-
bld javadoc                [Windows]
@end example

@noindent
The JavaDoc files end up in the @code{work/javadoc} directory.


@section Deploying A Kiss Application

The only file needed to deploy the application is @code{Kiss.war} It
can be built by typing @code{./bld war} at a command prompt.
@code{Kiss.war} ends up in the @code{work} directory.  If you
have your IDE create the @code{Kiss.war} file, it will likely not
work.  The @emph{Kiss} system requires a special build process because
application files are distributed in source form.  Therefore, @code{bld}
should be used to create the production WAR file.

If using @emph{Tomcat}, @code{Kiss.war} should be placed in the
@code{webapps} directory.  When @emph{Tomcat} starts, it will see the
file, unpack it, and run it.  The application will be available at
@code{[HOST]/Kiss}

Renaming @code{Kiss.war} to @code{ABC.war}, for example, will cause
the application path to change to @code{[HOST]/ABC}

@section Learning The System

In order to start getting a feel for how @code{Kiss} applications
function, in terms of the back-end, look at files in the
@code{src/main/backend/services} directory.  With @code{Kiss} you can
develop applications in several different languages.  The @code{services}
example shows the same code in all of the supported languages.

In terms of the front-end, see the example files under @code{src/main/frontend/page1}
