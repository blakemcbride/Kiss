@c -*-texinfo-*-

@c  Copyright (c) 2018 Blake McBride
@c  All rights reserved.
@c
@c  Redistribution and use in source and binary forms, with or without
@c  modification, are permitted provided that the following conditions are
@c  met:
@c
@c  1. Redistributions of source code must retain the above copyright
@c  notice, this list of conditions and the following disclaimer.
@c
@c  2. Redistributions in binary form must reproduce the above copyright
@c  notice, this list of conditions and the following disclaimer in the
@c  documentation and/or other materials provided with the distribution.
@c
@c  THIS SOFTWARE IS PROVIDED BY THE COPYRIGHT HOLDERS AND CONTRIBUTORS
@c  "AS IS" AND ANY EXPRESS OR IMPLIED WARRANTIES, INCLUDING, BUT NOT
@c  LIMITED TO, THE IMPLIED WARRANTIES OF MERCHANTABILITY AND FITNESS FOR
@c  A PARTICULAR PURPOSE ARE DISCLAIMED. IN NO EVENT SHALL THE COPYRIGHT
@c  HOLDER OR CONTRIBUTORS BE LIABLE FOR ANY DIRECT, INDIRECT, INCIDENTAL,
@c  SPECIAL, EXEMPLARY, OR CONSEQUENTIAL DAMAGES (INCLUDING, BUT NOT
@c  LIMITED TO, PROCUREMENT OF SUBSTITUTE GOODS OR SERVICES; LOSS OF USE,
@c  DATA, OR PROFITS; OR BUSINESS INTERRUPTION) HOWEVER CAUSED AND ON ANY
@c  THEORY OF LIABILITY, WHETHER IN CONTRACT, STRICT LIABILITY, OR TORT
@c  (INCLUDING NEGLIGENCE OR OTHERWISE) ARISING IN ANY WAY OUT OF THE USE
@c  OF THIS SOFTWARE, EVEN IF ADVISED OF THE POSSIBILITY OF SUCH DAMAGE.



@node Introduction

@html
@include style.css
@end html

@chapter Introduction
@c @pdfchapter{Introduction}

The @emph{KISS Framework} is an application development framework for
developing web-based business applications, portable desktop
applications, and command-line utilities.  The main home for
@emph{Kiss} is @uref{https://github.com/blakemcbride/Kiss}


Kiss' focus is on simplicity and development speed.  By being simple
to develop in, the development and support of the application can occur
more rapidly.  Simplicity is achieved by abstracting away as much
common functionality as possible so that developer lines of code are
maximally applicable to the application solution rather than
infrastructure and support of the framework.  Throughout the
framework, business-normal defaults have been employed in order to
minimize commonly needed functionality.

Another goal of the @emph{Kiss} framework is to be a complete web-based
application development solution.  @emph{Kiss} isn't a browser solution
alone, nor is it a back-end solution.  @emph{Kiss} includes solutions for
both ends -- although the two sides may largely be used independently.

@emph{Kiss} attempts to create a consistent interface.  This can greatly
simplify code even in simple cases.  For example, in terms of an input
text control, why would you disable/enable it with:

@example
$('#id').prop('disable', false);
$('#id').prop('disable', true);
@end example

and then hide/show it with:

@example
$('#id').hide();
$('#id').show();
@end example

@emph{Kiss} provides a consistent interface.  With @emph{Kiss}, you would do:

@example
$$('id').disable();
$$('id').enable();
$$('id').hide();
$$('id').show();
@end example
    
@emph{Kiss} is designed to be simple to get started with, simple to learn,
and simple to use.  @emph{Kiss} does this while supporting important
technologies such as micro-services, front-end components, and SQL.

The term @emph{single page application} has several, subtly different
meanings.  One meaning is that the entire application code is bundled
into a single file or HTTP GET request.  In that sense, @emph{Kiss} is not a
single page application.  This makes no sense for a business
application that could have hundreds of screens.

Another meaning of the term @emph{single page application} is that there is
only a single @code{html} tag and all of the remaining pages are
modifications of the original @code{html} tag contents.  In this
sense, @emph{Kiss} is a @emph{single page application}.  @emph{Kiss}
applications lazy-load as needed.  Browser cache is leveraged to
minimize Internet traffic.

@emph{Kiss} is used in a production environment and built by someone with
more than 30 years of experience as a framework designer and a business
application software engineer.  So @emph{Kiss} is not a proof of concept.

@emph{Kiss} was built as a solution to the challenges faced by the author
when developing web-based business applications.  As such, @emph{Kiss} is
more a solution for business application development than for the
development of public facing company presentation web sites.

Another goal of @emph{Kiss} is to keep the front-end and back-end as
independent of each other as possible.  To this end, communications
between the front-end and back-end occur via REST services and JSON.
This accomplishes two things.  First, it allows your organization to be
best prepared for the ever-evolving software environment.  Pieces can
be changed and enhanced without causing massive rewrites of the entire
system.  The second advantage is that by pushing as much processing to
the front-end as possible (rendering the display on the front-end),
the system can better scale.

@section Writing Sophisticated Applications

@emph{Kiss} comes as a very basic running application. Since the
front-end and back-end can be changed while the system is running, it
is intended that a developer would start with this basic application
and build it into the desired application.

The basic application the system comes with does not represent the
kind of application that can be built with @emph{Kiss}. It is
intentionally kept basic so that it is easy to see how it works.  In a
real application, all of the possibilities made available with
@emph{HTML} and @emph{CSS} are at your disposal without limitation. In
other words, the only limitations to the sophistication of the
application are your imagination and the amount of work you're willing
to put into it.

@emph{Kiss} comes with a complete front-end framework that includes
most of the things needed for a modern business web application. In
fact, it is currently being used as the sole framework in some very
large production systems.  It works well. However, there is nothing in
@emph{Kiss} that prevents you from using other front-end frameworks such as
@emph{React}, @emph{Angular}, @emph{Vue}, etc. In fact, except for
dealing with web services and authentication, you could exclusively
use your preferred front-end framework.

The @emph{Kiss} back-end is a complete back-end. It is currently
being used by large web applications involving many hundreds of
screens and SQL tables and thousands of users. The @emph{Kiss}
back-end contains many built-in features to efficiently handle many,
many simultaneous users. Like the front-end, the @emph{Kiss} back-end
is designed to be as simple as possible to build production-quality
web applications. For example, a web service can be built with a single
source file containing just a handful of lines of code. Everything in
@emph{Kiss} is designed to be simple to understand, simple to use, and
yet be fully compiled and run at top speed.

@section @emph{Kiss} Highlights

Some highlights of the @emph{Kiss} system include:

@subsection Back-end Highlights

@enumerate
@item
Micro services - add, change, or delete a web service on a running system.
@item
Each web method is in a single file and is very simple.  No
configuration files or setup code are needed.
@item
Easy access to common SQL databases with support for nested queries
without cursor interference.
@item
All REST services are stateless.  However, the system fully
authenticates each request.
@item
Changes to web services occur immediately on a running system,
without the need to reboot the application.
@item
A growing class library to handle common business application needs.
@item
The back-end framework is written in Java, and the system is portable to
Linux or Windows servers.
@item
Web services may be written in Groovy, Java or Common Lisp.  Python,
JavaScript, Ruby, and Scala are expected to follow soon.
@item
User authentication
@item
Asynchronous back-end REST services (via a queue and thread pool)
provide support for heavy loads and high throughput.
@item
A powerful and convenient class library for dealing with SQL persistence that
supports PostgreSQL, Microsoft SQL Server, MySQL, Oracle, and SQLite.
@item
Built-in interface to the @emph{Ollama} LLM server used to add AI features to your application.
@end enumerate


@subsection Front-end Highlights

@enumerate
@item
Build your own HTML components, thus encapsulating any amount of code
into a simple, custom HTML tag.
@item
Browser cache control.  Never ask your users to clear their browser cache again.
@item
All code is written in JavaScript/HTML/CSS.  No need for a complex build
and debug process, nor any need to learn yet another language.
@item
Growing list of included business oriented components designed to
provide simple access to fully functional business components.
@item
Straight forward means of designing your own components without a lot
of hidden and unpredictable magic.
@item
System is small and concise, rather than hundreds of megabytes other
systems take up.
@item
Consistent and simple API.
@end enumerate

@subsection Back-end Web Service Example


The following example depicts a complete back-end web service.  The
path to the file is its URL.  The class name is the web method name.

The file is a text file, but compiled code gets executed.
Authentication occurs before @code{main} is called.

Simply drop the file in place and the web service and method become
immediately available on a running system.  Changes to the service
take effect immediately (no need to reboot the server app).  There are
no configuration files or other code that needs to be changed.

For example, the following file is located in the
@code{services} directory.

@example
class MyWebService @{
    void myWebMethod(JSONObject injson, JSONObject outjson) @{
        int num1 = injson.getInt("num1");
        int num2 = injson.getInt("num2");
        outjson.put("result", num1 + num2);
    @}
@}
@end example

@subsection Front-end Web Service Usage Example

The following front-end example utilizes the web service defined in the
previous sub-section.

@example
let data = @{
    num1:  22,
    num2:  11
@};
let res = await Server.call("services/MyWebService", "myWebMethod", data);
if (res._Success) @{
    let result = res.result;
    //...
@}
@end example
    

@section Supported Environments

@subsection Development Environment

The following development platforms are supported by the @emph{Kiss}
framework:

@itemize @bullet
@item
Linux
@item
macOS
@item
Windows
@item
WSL under Windows
@item
BSD
@item
OpenIndiana@footnote{The SQLite interface does not work on OpenIndiana due to reasons unrelated to OpenIndiana.
However, @emph{Kiss} works fine with the other supported databases on OpenIndiana.  Also, the demo that @emph{Kiss} comes with will not
run completely on OpenIndiana because it depends on SQLite.  Once another database is configured, all is well.}
@item
Haiku@footnote{The SQLite interface does not work on Haiku due to reasons unrelated to Haiku.
However, @emph{Kiss} works fine with the other supported databases on Haiku.  Also, the demo that @emph{Kiss} comes with will not
run completely on Haiku because it depends on SQLite.  Once another database is configured, all is well.}
@end itemize

@subsection Production Environment


The following production platforms are supported by the @emph{Kiss}
framework:

@itemize @bullet
@item
Linux
@item
Windows Server
@item
BSD
@item
OpenIndiana
@end itemize


@subsection Databases Supported


The following database servers are supported by the @emph{Kiss}
framework:

@itemize @bullet
@item
PostgreSQL
@item
Microsoft SQL Server
@item
MySQL
@item
Oracle
@item
SQLite
@end itemize


@subsection Java

The system is tested with Java versions 17, and 21.  Any Java version above 17 
is expected to work.

@section REST vs. JSON-RPC

Throughout this manual and the entire @emph{Kiss} system, the
communications mechanism between the front-end and back-end is referred to as
@emph{REST}.  In fact, this is not true.  The truth is that all of the
communications between the back-end and front-end are more correctly
termed @emph{JSON-RPC}. @footnote{@emph{RPC} stands for @emph{Remote Procedure Call}.}
The reasons for this misnomer are as follows:

@itemize @bullet
@item
The term @emph{REST} is far better understood in the industry.
@item
The difference between the two is small.  
@item
They both send and receive data over standard @emph{HTTP} calls.
@item
@emph{JSON-RPC} is built on a subset of @code{REST}.
@end itemize

The following describes the differences between @emph{REST} and @emph{JSON-RPC}:

@enumerate
@item 
@emph{REST} is more @emph{CRUD} oriented; @emph{JSON-RPC} is more like a generic remote function call.
@item
There are two types of errors that can occur when the front-end calls a back-end service as follows:
@table @code
@item Internet communications error
Were the front-end and back-end able to communicate, or was there some sort of Internet issue?
@item Operation failure
The front-end and back-end were able to communicate, but the requested operation (like updating an employee record) failed for some application-specific reason.
@end table
@emph{REST} utilizes standard @emph{HTTP} return codes to indicate all success or failure operations.  @emph{JSON-RPC} utilizes standard @code{HTTP} return codes to indicate the success or failure of the Internet communications, but
error codes for the operations are returned as part of the returned value rather than an @code{HTTP} error code.  

An operation may fail for nearly an infinite number of operation-specific reasons.  Trying to map these reasons into a fixed set of standard @emph{HTTP} error codes is not possible.  Therefore, in essence, the @emph{JSON-RPC} call
uses the @emph{HTTP} error codes to indicate communications failures but uses internal operation-specific error codes to indicate operation failures.

Therefore, if the communication between the front-end and back-end succeeded but the operation failed, @emph{JSON-RPC} would return an @emph{HTTP} success code.  The operation error would be returned as part of the return value.
@item
@emph{REST} utilizes standard @emph{HTTP} operations such as
@emph{GET}, @emph{PUT}, @emph{DELETE}, @emph{POST}, etc.  The problem
with this is that if you issue a @emph{GET} operation to get some data
and the data subsequently changes because of some other user, and then
you issue another @emph{GET} operation, the @emph{REST} call will return the old, cached data.

@emph{JSON-RPC} always uses the @emph{POST} call for all operations so that no data is cached on the front-end.  This means a call to get the same data will return the latest data rather than the old, cached data.

@item
@emph{REST} is limited to the operations predefined by the @emph{HTTP} standards.  @code{JSON-RPC} can use arbitrary, application-defined operations.
@end enumerate





@section HTML component usage
   
To use a component, add to HTML:

@example
<my-component></my-component>
@end example

Add to JavaScript:

@example
Utils.useComponent('MyComponent');
@end example
    
The component can put any HTML in the component location, have any
functionality, have its own modal windows, and use other components.
The component can have custom and non-custom attributes (like style).
Non-custom attributes do what you'd expect them to do.

The system also supports tag-less components.  This provides an easy
way to package arbitrary blocks of code (that can have screens too).

@section System Maturity And Stability

The Kiss system has been used in production environments for several
years.  Additionally, several commercial applications utilize @emph{Kiss}.
In spite of this, however, @emph{Kiss} is constantly being adjusted
in response to additional needs, evolving environments, and bug fixes.

We use @emph{Kiss} daily in a Linux and PostgreSQL environment.
Therefore, it is best tested there.  While we support all of the listed
environments, they receive a bit less testing.  If you encounter a problem,
please reach out to us.  It is probably easy for us to fix, and we
are happy to do so.


@section Getting All Source Code

Source code for all of @emph{Kiss} and its dependencies is freely
available.  The builder program located at
@code{src/main/precompiled/Tasks.java} contains the paths
to all of the external dependencies (those not included in the
@emph{Kiss} distribution).  The following lists the paths to the
internal dependencies (those included with @emph{Kiss}):

@table @code
@item abcl.jar
@url{https://common-lisp.net/project/armedbear}
@item SimpleWebServer.jar (only used during development)
@url{https://github.com/blakemcbride/SimpleWebServer}
@end table

@section Support, Contact, And Links

@indent
The @emph{Kiss} main website is at @uref{https://kissweb.org}

Source code is at @uref{https://github.com/blakemcbride/Kiss}

Public discussion and support are available at @* 
@uref{https://github.com/blakemcbride/Kiss/discussions}

Issue tracking is at @uref{https://github.com/blakemcbride/Kiss/issues}

Commercial support is available.  Contact us via email at @email{blake@@mcbridemail.com}

@section License

Copyright (c) 2018 Blake McBride (blake@@mcbridemail.com)

Permission is hereby granted, free of charge, to any person obtaining
a copy of this software and associated documentation files (the
``Software''), to deal in the Software without restriction, including
without limitation the rights to use, copy, modify, merge, publish,
distribute, sublicense, and/or sell copies of the Software, and to
permit persons to whom the Software is furnished to do so, subject to
the following conditions:

1. Redistributions of source code must retain the above copyright
notice, this list of conditions, and the following disclaimer.

2. Redistributions in binary form must reproduce the above copyright
notice, this list of conditions and the following disclaimer in the
documentation and/or other materials provided with the distribution.

THIS SOFTWARE IS PROVIDED BY THE COPYRIGHT HOLDERS AND CONTRIBUTORS
``AS IS'' AND ANY EXPRESS OR IMPLIED WARRANTIES, INCLUDING, BUT NOT
LIMITED TO, THE IMPLIED WARRANTIES OF MERCHANTABILITY AND FITNESS FOR
A PARTICULAR PURPOSE ARE DISCLAIMED. IN NO EVENT SHALL THE COPYRIGHT
HOLDER OR CONTRIBUTORS BE LIABLE FOR ANY DIRECT, INDIRECT, INCIDENTAL,
SPECIAL, EXEMPLARY, OR CONSEQUENTIAL DAMAGES (INCLUDING, BUT NOT
LIMITED TO, PROCUREMENT OF SUBSTITUTE GOODS OR SERVICES; LOSS OF USE,
DATA, OR PROFITS; OR BUSINESS INTERRUPTION) HOWEVER CAUSED AND ON ANY
THEORY OF LIABILITY, WHETHER IN CONTRACT, STRICT LIABILITY, OR TORT
(INCLUDING NEGLIGENCE OR OTHERWISE) ARISING IN ANY WAY OUT OF THE USE
OF THIS SOFTWARE, EVEN IF ADVISED OF THE POSSIBILITY OF SUCH DAMAGE.

@section Acknowledgments

The Kiss design, code, documentation, and web site were written by
Blake McBride.  The author gratefully acknowledges and appreciates,
among others, the following:

Apache Groovy located at @uref{https://groovy-lang.org}

JSON-Java located at @uref{https://github.com/stleary/JSON-java}
(although I am using a modified version available at
@uref{https://github.com/blakemcbride/JSON-java})

C3P0 located at @uref{https://www.mchange.com/projects/c3p0}

Texinfo located at @uref{https://www.gnu.org/software/texinfo/}

ABCL project located at @uref{https://common-lisp.net/project/armedbear}

Melaine Sarbey (@email{melswildart@@gmail.com}) for creating the Kiss logo.

