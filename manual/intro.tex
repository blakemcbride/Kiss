@c -*-texinfo-*-

@c  Copyright (c) 2018 Blake McBride
@c  All rights reserved.
@c
@c  Redistribution and use in source and binary forms, with or without
@c  modification, are permitted provided that the following conditions are
@c  met:
@c
@c  1. Redistributions of source code must retain the above copyright
@c  notice, this list of conditions and the following disclaimer.
@c
@c  2. Redistributions in binary form must reproduce the above copyright
@c  notice, this list of conditions and the following disclaimer in the
@c  documentation and/or other materials provided with the distribution.
@c
@c  THIS SOFTWARE IS PROVIDED BY THE COPYRIGHT HOLDERS AND CONTRIBUTORS
@c  "AS IS" AND ANY EXPRESS OR IMPLIED WARRANTIES, INCLUDING, BUT NOT
@c  LIMITED TO, THE IMPLIED WARRANTIES OF MERCHANTABILITY AND FITNESS FOR
@c  A PARTICULAR PURPOSE ARE DISCLAIMED. IN NO EVENT SHALL THE COPYRIGHT
@c  HOLDER OR CONTRIBUTORS BE LIABLE FOR ANY DIRECT, INDIRECT, INCIDENTAL,
@c  SPECIAL, EXEMPLARY, OR CONSEQUENTIAL DAMAGES (INCLUDING, BUT NOT
@c  LIMITED TO, PROCUREMENT OF SUBSTITUTE GOODS OR SERVICES; LOSS OF USE,
@c  DATA, OR PROFITS; OR BUSINESS INTERRUPTION) HOWEVER CAUSED AND ON ANY
@c  THEORY OF LIABILITY, WHETHER IN CONTRACT, STRICT LIABILITY, OR TORT
@c  (INCLUDING NEGLIGENCE OR OTHERWISE) ARISING IN ANY WAY OUT OF THE USE
@c  OF THIS SOFTWARE, EVEN IF ADVISED OF THE POSSIBILITY OF SUCH DAMAGE.


@chapter Introduction
@c @pdfchapter{Introduction}

The @emph{KISS Framework} is an application development framework for
developing web-based business applications.  The main home for @emph{Kiss}
is @uref{https://github.com/blakemcbride/Kiss}


Kiss' focus is on
simplicity and development speed.  By being simple to develop in,
development and support of the application can occur more rapidly.
Simplicity is achieved by abstracting away as much common
functionality as possible so that developer lines of code are
maximally applicable to the application solution rather than infrastructure and
support of the framework.  Throughout the framework, business-normal
defaults have been employed in order to minimize commonly needed
functionality.

Another goal of the @emph{Kiss} framework is to be a complete web-based
application development solution.  @emph{Kiss} isn't a browser solution
alone, nor is it a back-end solution.  @emph{Kiss} includes solutions for
both ends -- although the two sides may largely be used independently.

@emph{Kiss} attempts to create a consistent interface.  This can greatly
simply code even in simple cases.  For example, in terms of an input
text control, why would you disable/enable it with:

@example
$('#id').prop('disable', false);
$('#id').prop('disable', true);
@end example

and then hide/show it with:

@example
$('#id').hide();
$('#id').show();
@end example

@emph{Kiss} provides a consistent interface.  With @emph{Kiss}, you would do:

@example
$$('id').disable();
$$('id').enable();
$$('id').hide();
$$('id').show();
@end example
    
@emph{Kiss} is designed to be simple to get started with, simple to learn,
and simple to use.  @emph{Kiss} does this while supporting important
technologies such as micro-services, front-end components, and SQL.

The term @emph{single page application} has several, subtly different,
meanings.  One meaning is that the entire application code is bundled
into a single file or HTTP GET request.  In that sense, @emph{Kiss} is not a
single page application.  This makes no sense for a business
application that could have hundreds of screens.

Another meaning of the term @emph{single page application} is that there is
only a single @code{html} tag and all of the remaining pages are
modifications of the original @code{html} tag contents.  In this
sense, @emph{Kiss} is a @emph{single page application}.  @emph{Kiss}
applications lazy-load as needed.  Browser cache is leveraged to
minimize Internet traffic.

@emph{Kiss} is used in a production environment and built by someone with
more than 30 years experience as a framework designer and a business
application software engineer.  So @emph{Kiss} is not a proof of concept.

@emph{Kiss} was built as a solution to the challenges faced by the author
when developing web-based business applications.  As such, @emph{Kiss} is
more a solution for business application development than for the
development of public facing company presentation web sites.

Another goal of @emph{Kiss} is to keep the front-end and back-end as
independent of each other as possible.  To this end, communications
between the front-end and back-end occur via REST services and JSON.
This accomplished two things.  First, it allows your organization to be
best prepared for the ever-evolving software environment.  Pieces can
be changed and enhanced without causing massive re-writes of the entire
system.  The second advantage is that by pushing as much processing to
the front-end as possible (rendering the display on the front-end),
the system can better scale.

@section @emph{Kiss} Highlights

Some highlights of the @emph{Kiss} system include:

@subsection Back-end Highlights

@enumerate
@item
Micro services - add, change, or delete a web service on a running system.
@item
Each web-method is in a single file and are very simple.  No
configuration files or setup code needed.
@item
Easy access to common SQL databases with support for nested queries
without cursor interference.
@item
All REST services are stateless.  However, the system fully
authenticates each request.
@item
Changes to web services occur immediately, on a running system,
without the need to reboot the application.
@item
A growing class library to handle common business application needs.
@item
Back-end framework is written in Java, and the system is portable to
Linux or Windows servers.
@item
Web services may be written in Groovy, Java or Common Lisp.  Python,
JavaScript, Ruby, and Scala are expected to follow soon.
@item
User authentication
@item
Asynchronous back-end REST services (via a queue and thread pool)
provide support for heavy loads and high throughput.
@item
A powerful and convenient class library for dealing with SQL persistence that
supports PostgreSQL, Microsoft SQL Server, MySQL, Oracle, and SQLite.
@end enumerate

@subsection Front-end Highlights

@enumerate
@item
Build your own HTML components thus encapsulating any amount of code
into a simple, custom HTML tag.
@item
Browser cache control.  Never ask your users to clear their browser cache again.
@item
All code written in JavaScript/HTML/CSS.  No need for a complex build
and debug process, nor any need to learn yet another language.
@item
Growing list of included business oriented components designed to
provide simple access to fully functional business components.
@item
Straight forward means of designing your own components without a lot
of hidden and unpredictable magic.
@item
System is small and concise, rather then hundreds of megabytes other
systems take up.
@item
Consistent and simple API.
@end enumerate

@subsection Back-end Web Service Example


The following example depicts a complete back-end web service.  The
path to the file is its URL.  The class name is the web method name.

The file is a text file, but compiled code gets executed.
Authentication occurs before @code{main} is called.

Simply drop the file in place and the web service and method become
immediately available on a running system.  Changes to the service
take effect immediately (no need to reboot the server app).  There are
no configuration files or other code that needs to be changed.

For example, the following file is located in the
@code{services} directory.

@example
class MyWebService @{
    void myWebMethod(JSONObject injson, JSONObject outjson) @{
        int num1 = injson.getInt("num1");
        int num2 = injson.getInt("num2");
        outjson.put("result", num1 + num2);
    @}
@}
@end example

@subsection Front-end Web Service Usage Example

The following front-end example utilizes the web service defined in the
previous sub-section.

@example
let data = @{
    num1:  22,
    num2:  11
@};
let res = await Server.call("services/MyWebService", "myWebMethod", data);
if (res._Success) @{
    let result = res.result;
    //...
@}
@end example
    
@section HTML component usage
   
To use a component add to HTML:

@example
<my-component></my-component>
@end example

Add to JavaScript:

@example
Utils.useComponent('MyComponent');
@end example
    
The component can put any HTML in the component location, have any
functionality, have its own modal windows, and use other components.
The component can have custom and non-custom attributes (like style).
Non-custom attributes do what you's expect them to do.

The system also supports tag-less components.  This provides an easy
way to package arbitrary blocks of code (that can have screens too).

@section System Maturity And Future

Largely, the Kiss system, in the form of running applications, has
been used in production environments for a few years.  The effort to
tease away the generic parts and generalize them is recent (2018).
Moving forward, as time permits, the priorities are as follows:

@enumerate
@item
Improve documentation
@item
More examples
@item
Greater front-end functionality (modal windows, grid control, etc. - note that these can be done by any other known facility.  They just don't come included with Kiss yet.)
@item
Support more back-end languages (such as Python, JavaScript, Ruby, etc.)
@end enumerate

@section Getting All Source Code

Source code for all of @emph{Kiss} and its dependencies is freely
available.  The builder program located at
@code{src/main/core/org/kissweb/builder/Tasks.java} contains the paths
to all of the external dependencies (those not included in the
@emph{Kiss} distribution).  The following lists the paths to the
internal dependencies (those included with @emph{Kiss}):

@table @code
@item abcl.jar
@url{https://common-lisp.net/project/armedbear}
@item json.jar
@url{https://github.com/blakemcbride/JSON-java}
@item SimpleWebServer.jar (only used during development)
@url{https://github.com/blakemcbride/SimpleWebServer}
@end table

@section Contact And Links

@indent
The @emph{Kiss} main web site is at @uref{http://kissweb.org}

Source code is at @uref{https://github.com/blakemcbride/Kiss}

Public discussion and support is available at @* @uref{https://groups.google.com/forum/#!forum/kissweb}

Issue tracking is at @uref{https://github.com/blakemcbride/Kiss/issues}

Commercial support is available.  Contact us via email at @email{kissweb.org@@gmail.com}

@section License

Copyright (c) 2018 Blake McBride (blake@@mcbridemail.com)

Permission is hereby granted, free of charge, to any person obtaining
a copy of this software and associated documentation files (the
``Software''), to deal in the Software without restriction, including
without limitation the rights to use, copy, modify, merge, publish,
distribute, sublicense, and/or sell copies of the Software, and to
permit persons to whom the Software is furnished to do so, subject to
the following conditions:

1. Redistributions of source code must retain the above copyright
notice, this list of conditions, and the following disclaimer.

2. Redistributions in binary form must reproduce the above copyright
notice, this list of conditions and the following disclaimer in the
documentation and/or other materials provided with the distribution.

THIS SOFTWARE IS PROVIDED BY THE COPYRIGHT HOLDERS AND CONTRIBUTORS
``AS IS'' AND ANY EXPRESS OR IMPLIED WARRANTIES, INCLUDING, BUT NOT
LIMITED TO, THE IMPLIED WARRANTIES OF MERCHANTABILITY AND FITNESS FOR
A PARTICULAR PURPOSE ARE DISCLAIMED. IN NO EVENT SHALL THE COPYRIGHT
HOLDER OR CONTRIBUTORS BE LIABLE FOR ANY DIRECT, INDIRECT, INCIDENTAL,
SPECIAL, EXEMPLARY, OR CONSEQUENTIAL DAMAGES (INCLUDING, BUT NOT
LIMITED TO, PROCUREMENT OF SUBSTITUTE GOODS OR SERVICES; LOSS OF USE,
DATA, OR PROFITS; OR BUSINESS INTERRUPTION) HOWEVER CAUSED AND ON ANY
THEORY OF LIABILITY, WHETHER IN CONTRACT, STRICT LIABILITY, OR TORT
(INCLUDING NEGLIGENCE OR OTHERWISE) ARISING IN ANY WAY OUT OF THE USE
OF THIS SOFTWARE, EVEN IF ADVISED OF THE POSSIBILITY OF SUCH DAMAGE.

@section Acknowledgments

The Kiss design, code, documentation, and web site were written by
Blake McBride.  The author gratefully acknowledges and appreciates,
among others, the following:


Apache Groovy located at @uref{http://groovy-lang.org}

Dynamic Loader located at @uref{https://github.com/dvare/dynamic-loader}

JSON-Java located at @uref{https://github.com/stleary/JSON-java}
(although I am using a modified version available at
@uref{https://github.com/blakemcbride/JSON-java})

C3P0 located at @uref{https://www.mchange.com/projects/c3p0}

Texinfo located at @uref{https://www.gnu.org/software/texinfo/}

ABCL project located at @uref{https://common-lisp.net/project/armedbear}

Melaine Sarbey (@email{melswildart@@gmail.com}) for creating the Kiss logo.

