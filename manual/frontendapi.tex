@c -*-texinfo-*-

@c  Copyright (c) 2018 Blake McBride
@c  All rights reserved.
@c
@c  Redistribution and use in source and binary forms, with or without
@c  modification, are permitted provided that the following conditions are
@c  met:
@c
@c  1. Redistributions of source code must retain the above copyright
@c  notice, this list of conditions and the following disclaimer.
@c
@c  2. Redistributions in binary form must reproduce the above copyright
@c  notice, this list of conditions and the following disclaimer in the
@c  documentation and/or other materials provided with the distribution.
@c
@c  THIS SOFTWARE IS PROVIDED BY THE COPYRIGHT HOLDERS AND CONTRIBUTORS
@c  "AS IS" AND ANY EXPRESS OR IMPLIED WARRANTIES, INCLUDING, BUT NOT
@c  LIMITED TO, THE IMPLIED WARRANTIES OF MERCHANTABILITY AND FITNESS FOR
@c  A PARTICULAR PURPOSE ARE DISCLAIMED. IN NO EVENT SHALL THE COPYRIGHT
@c  HOLDER OR CONTRIBUTORS BE LIABLE FOR ANY DIRECT, INDIRECT, INCIDENTAL,
@c  SPECIAL, EXEMPLARY, OR CONSEQUENTIAL DAMAGES (INCLUDING, BUT NOT
@c  LIMITED TO, PROCUREMENT OF SUBSTITUTE GOODS OR SERVICES; LOSS OF USE,
@c  DATA, OR PROFITS; OR BUSINESS INTERRUPTION) HOWEVER CAUSED AND ON ANY
@c  THEORY OF LIABILITY, WHETHER IN CONTRACT, STRICT LIABILITY, OR TORT
@c  (INCLUDING NEGLIGENCE OR OTHERWISE) ARISING IN ANY WAY OUT OF THE USE
@c  OF THIS SOFTWARE, EVEN IF ADVISED OF THE POSSIBILITY OF SUCH DAMAGE.


@chapter Front-end API

The front-end API are all the facilities that run on the browser.
This includes HTML, CSS, JavaScript, image files, etc.  The @emph{Kiss}
back-end does not produce or modify HTML or JavaScript code.  These
files are served, unaltered, by the server as they are on the back-end
disk.  The @code{Kiss} model is that the browser receives these files
from the server, and that they include all the code that the browser
needs to perform its function.  Besides these static files, all data
is communicated between the back-end and front-end via REST services.

Having all of the display logic running on the front-end or user's
browser makes a lot of sense for the following reasons:

@enumerate
@item
Minimize the dependence the front-end and back-end have on each other.  This means that one end can be changed without necessitating
the need for the other to change.  In other words, they are minimally dependent on each other.
@enumerate a
@item
In this rapidly changing environment, minimizing dependencies means minimizing the amount of code that has to be changed as technology changes.
@item
Code is easier to understand and maintain since you don't have four totally different languages in the same file.
@end enumerate
@item
Push as much processing to the client side so that the back-end can scale easier.
@end enumerate

@noindent
All of this leads to the following:
@enumerate
@item
Shorter development time
@item 
Easier to maintain
@item
Be most prepared for future changes
@item
Reduces server costs
@item
Reduced development time and cost
@end enumerate

The front-end API is documented in the file @code{manual/jsdoc/index.html}.



@section Calling REST Services

On the front-end, the class @code{Server} is what deals with the REST
communications between the front-end and back-end.  In it, there are a
handful of methods that deal with the environment such as the back-end
URL.  All communications between the back-end and front-end are done
with JSON.

The way it works is the login service requires a user name and
password, and it returns a login token (uuid).  That token is used in
all future calls, and it gets automatically invalidated after a certain
amount of non-use time.  There is no state kept on the back-end.  Each
REST call must login to the back-end with the provided token in order
to be authenticated to communicate.

The method used to communicate is named @code{Server.call()}. It is passed the
path to the REST service, the REST service method name, and a JSON
object that is send to the back-end method.  A @code{Promise} is
returned that is used to obtain the result of the call.  This
can also be used with @code{async/await}.

There is also a @code{logout} method that simply erases the
login token so that future communications cannot occur.

The @code{Server} class also includes a method (@code{fileUploadSend})
that makes it easy to upload files.

@section Kiss Components


Although HTML provides what is needed for real applications, it
provides those facilities in too low a level to be useful without a
lot of custom code.  Custom components (tags) allow you to encapsulate
that advanced behavior into what is used as and appears as native
functionality.

@emph{Kiss} provides the ability to create your own custom HTML tags
or elements as well as use those provided by @emph{Kiss}.  There are
two principal user methods that make this work.
@code{Utils.useComponent} is used to load either a @emph{Kiss} defined
component or one you define yourself (there is no difference).  This
loads the JavaScript file that defines the new tag.  All of the
components that come with the @emph{Kiss} system are under the
@code{src/main/webapp/kiss/component} directory.  You can see those
files for examples of how custom tags are defined.

New application pages are loaded with the @code{Utils.loadPage}
method.  In addition to loading the HTML and JavaScript code
associated with that page, this method performs the processing
necessary to make the components work.  It does this intelligently so
that, for example, one component can use another component without any
special loading order requirement.

Briefly, the code that describes the custom tag must describe what the
tag is replaced with.  Ultimately, it must boil down to straight HTML,
CSS, and JavaScript code.


@subsection Tagless Components

Let's say you have a pop-up window that allows a user to search for employees, or product, or whatever.
The user gets a variety of search capabilities and the selected item is returned. Let's further say
that you need this functionality in several places within your application.  These are tagless
components.  They aren't placed with a custom tag.  They are a response to an event like a button push.
Tagless components allow you to encapsulate a block of functionality (including pop-up windows) into
a neat package that can be re-used in any number of places.

The method used to load tagless components is
@code{Utils.useTaglessComponent}.  Later, when the tagless component
is needed, one would execute @code{Kiss.MyComponent.run(in_data, on_exit)} (where
@code{MyComponent} is the name of your tagless component.  
@code{in_data} represents possible data passed to the component on entry.
@code{on_exit} is a function that gets executed when the component exits.
Arguments passed to @code{on_exit} are determined by the component.


@section Modal Popup Windows

@emph{Kiss} supports dragable, modal popup windows. @emph{HTML} is
used to describe the layout of the popup, and JavaScript is used to
control the appearance of the popup.

The @emph{HTML} portion is represented as a top-level @code{popup} tag
that represents the entire popup.  Withing the top-level @code{popup}
there most be two child tags named @code{popup-title} and
@code{popup-body}.  The first represents the single line header, and
the second represents the body of the popup window.

The top-level @code{popup} tag's attribute section must contain the following:

@table @code
@item id="my-popup"
an ID is needed to reference the popup
@item width="600px" height="300px"
Set the height and width of the popup
@end table

@noindent
An examples is as follows:

@example
<popup id="my-popup" width="600px" height="300px">
    <popup-title>The title</popup-title>
    <popup-body>
        The content
        <div style="display: inline-block; position: absolute; bottom: 20px; right: 20px;">
            <push-button id="cancel" style="margin-left: 15px;">Cancel</push-button>
            <push-button id="ok" style="margin-left: 15px;">Ok</push-button>
        </div>
    </popup-body>
</popup>
@end example

@noindent
There are only two JavaScript functions used to control the popup.

@table @code
@item Utils.popup_open(id [, focus-id])
open the popup indicated by the @code{id}, and if @code{focus-id} is present, set initial focus to that control
@item Utils.popup_close()
close the most recent popup
@end table

@noindent
Your defined responses to the buttons on the popup determines your dealing with the
data on the popup and when to close it.

@section File Uploads

@emph{Kiss} includes facilities that make it easy to upload a file or
multiple files.  The way to use it is as follows:

The HTML would contain two controls; a file input and a button.  The file
input looks as follows:

@example
<file-upload id="myUpload">Upload File</file-upload>
@end example

@noindent
The file input control allows the user to select the file or files to
be uploaded.  If multiple files are to be allowed the @code{multiple}
attribute should be added to the HTML.

The button is used to activate the upload process.  Your code that
sends the file(s) to the server should be attached to this button.

The @code{file-upload} control contains the helper functions
@code{numberOfUploadedFiles} and @code{getFormData}, and the main
function used to send the files is @code{fileUploadSend} in the
@code{Server} class.  The code would look like the following:

@example
   $$('upload').onclick(async () => @{
        if ($$('the-file').numberOfUploadFiles() < 1) @{
            Utils.showMessage('Error', 'You must first select a file to upload.');
            return;
        @}
        const fd = $$('the-file').getFormData();
        let data = @{
            var1: 22,    // just some random data we want to sent to the back-end
            var2: 33
        @}
        const r = await Server.fileUploadSend('theService, 'theMethod', fd, data);
    @});
@end example

@noindent
Back-end code would look like this:

@example
public void theMethod(JSONObject inJson, JSONObject outJson, MainServlet servlet) throws Exception @{
    String var1 = inJson.getString("var1");
    String var2 = inJson.getString("var2");
    if (servlet.getUploadFileCount() == 0)
        throw new Exception("No file specified.");
        
    String originalFileName = servlet.getUploadFileName(0);
    
    
    BufferedInputStream bis = servlet.getUploadBufferedInputStream(0);
    // do something with the file stream
    bis.close();
    
         -or-
         
    String localFileName = servlet.saveUploadFile(0);
@}
@end example




@section Utilities

@emph{Kiss} includes an ever-growing set of utilities to help deal with common tasks.
These utilities are located under the @code{src/main/webapp/kiss} directory and have names such as
@code{DateTimeUtils.js}, @code{DateUtils}, @code{TimeUtils}, @code{Utils}, etc.  These utilities
are documented in the front-end API documentation.

@section Controlling Browser Cache

Browser's have a mind of their own in term of deciding when to use
their cache for a file and when to download a new one.  This can cause
no end of trouble when code gets changed.  Some user files end up
being old from the browser cache, and others are freshly downloaded.
The old and new files don't agree with each other and all sorts of
errors occur.

@emph{Kiss} includes a facility to assure that all files are
downloaded afresh whenever the application changes while still taking
maximal advantage of the browser cache when the files have not
changed.  The only cost for this capability is the requirement
that the @code{index.html} file @emph{always} gets loaded afresh.
To that end, @emph{Kiss} has code to ignore browser cache and always load
@code{index.html} afresh.

@code{index.html} contains two variables names @code{softwareVersion} and @code{controlCache}.
Assuming @code{controlCache} is @code{true}, @emph{Kiss} has code that forces the browser
to re-load all files whenever @code{softwareVersion} changes.  After the code is re-loaded,
the browser cache will work as normal to maximally cache the files until the next
@code{softwareVersion} change.

@section Additional Resources

Although not a part of the @emph{Kiss} system, there are some very valuable technologies and libraries that
have been used with @emph{Kiss} in order to create some very powerful solutions.

The first is the Lovefield library that adds SQL capabilities on the browser side.  Data is persisted on the user's browser
and remains through browser or machine re-boots.  The library is located at @uref{https://github.com/google/lovefield}

A recent technology that has been used to enable browser application that run when there is no Internet connection
is called @emph{Service Workers}.

