@c -*-texinfo-*-

@c  Copyright (c) 2018 Blake McBride
@c  All rights reserved.
@c
@c  Redistribution and use in source and binary forms, with or without
@c  modification, are permitted provided that the following conditions are
@c  met:
@c
@c  1. Redistributions of source code must retain the above copyright
@c  notice, this list of conditions and the following disclaimer.
@c
@c  2. Redistributions in binary form must reproduce the above copyright
@c  notice, this list of conditions and the following disclaimer in the
@c  documentation and/or other materials provided with the distribution.
@c
@c  THIS SOFTWARE IS PROVIDED BY THE COPYRIGHT HOLDERS AND CONTRIBUTORS
@c  "AS IS" AND ANY EXPRESS OR IMPLIED WARRANTIES, INCLUDING, BUT NOT
@c  LIMITED TO, THE IMPLIED WARRANTIES OF MERCHANTABILITY AND FITNESS FOR
@c  A PARTICULAR PURPOSE ARE DISCLAIMED. IN NO EVENT SHALL THE COPYRIGHT
@c  HOLDER OR CONTRIBUTORS BE LIABLE FOR ANY DIRECT, INDIRECT, INCIDENTAL,
@c  SPECIAL, EXEMPLARY, OR CONSEQUENTIAL DAMAGES (INCLUDING, BUT NOT
@c  LIMITED TO, PROCUREMENT OF SUBSTITUTE GOODS OR SERVICES; LOSS OF USE,
@c  DATA, OR PROFITS; OR BUSINESS INTERRUPTION) HOWEVER CAUSED AND ON ANY
@c  THEORY OF LIABILITY, WHETHER IN CONTRACT, STRICT LIABILITY, OR TORT
@c  (INCLUDING NEGLIGENCE OR OTHERWISE) ARISING IN ANY WAY OUT OF THE USE
@c  OF THIS SOFTWARE, EVEN IF ADVISED OF THE POSSIBILITY OF SUCH DAMAGE.


@chapter Front-end API

The front-end API are all the facilities that run on the browser.
This includes HTML, CSS, JavaScript, image files, etc.  The @emph{Kiss}
back-end does not produce or modify HTML or JavaScript code.  These
files are served, unaltered, by the server as they are on the back-end
disk.  The @code{Kiss} model is that the browser receives these files
from the server, and that they include all the code that the browser
needs to perform its function.  Besides these static files, all data
is communicated between the back-end and front-end via REST services.

Having all of the display logic running on the front-end or user's
browser makes a lot of sense for the following reasons:

@enumerate
@item
Minimize the dependence the front-end and back-end have on each other.  This means that one end can be changed without necessitating
the need for the other to change.  In other words, they are minimally dependent on each other.
@enumerate a
@item
In this rapidly changing environment, minimizing dependencies means minimizing the amount of code that has to be changed as technology changes.
@item
Code is easier to understand and maintain since you don't have four totally different languages in the same file.
@end enumerate
@item
Push as much processing to the client side so that the back-end can scale easier.
@end enumerate

@noindent
All of this leads to the following:
@enumerate
@item
Shorter development time
@item 
Easier to maintain
@item
Be most prepared for future changes
@item
Reduces server costs
@item
Reduced development time and cost
@end enumerate



@section Calling REST Services

@section Kiss Components

@subsection Custom Tags

@subsection Tagless Components

@section Utilities

@section Controlling Browser Cache

@section Additional Resources

Although not a part of the @emph{Kiss} system, there are some very valuable technologies and libraries that
have been used with @emph{Kiss} in order to create some very powerful solutions.

The first is the Lovefield library that adds SQL capabilities on the brower side.  Data is persisted on the user's browser
and remains through browser or machine re-boots.  The library is located at @uref{https://github.com/google/lovefield}

A recent technology that has been used to enable browser application that run when there is no Internet connection
is calles @emph{Service Workers}.

